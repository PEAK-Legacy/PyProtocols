% Complete documentation on the extended LaTeX markup used for Python
% documentation is available in ``Documenting Python'', which is part
% of the standard documentation for Python.  It may be found online
% at:
%
%     http://www.python.org/doc/current/doc/doc.html

\documentclass{manual}

\title{Component Adaptation And Open Protocols}

\author{Phillip J. Eby and Tyler C. Sarna}

% Please at least include a long-lived email address;
% the rest is at your discretion.
%\authoraddress{
%	Organization name, if applicable \\
%	Street address, if you want to use it \\
%	Email: \email{transwarp@eby-sarna.com}
%}

\date{May 1, 2003}       % update before release!

%\release{0.5a1}   % release version; this is used to define the
                  % \version macro

%\makeindex          % tell \index to actually write the .idx file
%\makemodindex       % If this contains a lot of module sections.


\begin{document}

\maketitle

% This makes the contents more accessible from the front page of the HTML.
%\ifhtml
%\chapter*{Front Matter\label{front}}
%\fi

%\input{copyright}

\begin{abstract}

\noindent

The Python Protocols package provides framework developers and users with
tools for defining, declaring, and adapting components between interfaces,
even when those interfaces are defined using different mechanisms.

\end{abstract}

\tableofcontents

\chapter{Reference}

\section{\module{protocols} ---
         Protocol Definition, Declaration, and Adaptation}
\declaremodule{}{protocols}
\moduleauthor{Phillip J. Eby}{pje@telecommunity.com}
\sectionauthor{Phillip J. Eby}{pje@telecommunity.com}
\modulesynopsis{Protocol declaration and adaptation functions as described in
    PEP XXX.}

Protocols are blah blah blah...

(note: sections aren't necessarily in order yet...)

\subsection{Defining Interfaces and Protocols}
subsetting, extending...

\subsection{The Adaptation Protocol}

\subsection{Protocol Implication}

\subsection{Declaring Protocols for Classes}

\subsection{Module Contents}

\subsection{Examples}
Replacing introspection with Adaptation, simple adaptation...?

\subsection{Advanced Techniques}
\subsubsection{Instance-specific Declarations}
\subsubsection{Custom Declaration Mechanisms}
\subsubsection{Adapting ``Foreign'' Interfaces}



%\appendix
%\chapter{...}

%My appendix.

%The \code{\e appendix} markup need not be repeated for additional
%appendices.








%
%  The ugly "%begin{latexonly}" pseudo-environments are really just to
%  keep LaTeX2HTML quiet during the \renewcommand{} macros; they're
%  not really valuable.
%
%  If you don't want the Module Index, you can remove all of this up
%  until the second \input line.
%
%begin{latexonly}
\renewcommand{\indexname}{Module Index}
%end{latexonly}
%\input{mod\jobname.ind}     % Module Index

%begin{latexonly}
\renewcommand{\indexname}{Index}
%end{latexonly}
%\input{\jobname.ind}        % Index

\end{document}
