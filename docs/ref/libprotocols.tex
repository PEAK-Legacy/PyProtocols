\section{\module{protocols} ---
         Protocol Definition, Declaration, and Adaptation}
\declaremodule{}{protocols}
\moduleauthor{Phillip J. Eby}{pje@telecommunity.com}
\sectionauthor{Phillip J. Eby}{pje@telecommunity.com}
\modulesynopsis{Protocol declaration and adaptation functions as described in
    PEP XXX.}

\begin{quotation}
The typical Python programmer is an integrator, someone who is
connecting components from various vendors.  Often times the
interfaces between these components require an intermediate
adapter.  Usually the burden falls upon the programmer to
study the interface exposed by one component and required by
another, determine if they are directly compatible, or develop
an adapter.  Sometimes a vendor may even include the
appropriate adapter, but then searching for the adapter and
figuring out how to deploy the adapter takes time.

\hfill --- Martelli \& Evans, PEP 246
\end{quotation}

This package builds on the object adaptation protocol presented in \pep{246}
to make it easier for component authors, framework suppliers, and other
developers to:

\begin{itemize}

\item Specify what behavior a component requires or provides

\item Specify how to adapt the interface provided by one component to that
required by another

\item Specify how to adapt objects of a particular type or class (even
built-in types) to a particular required interface

\item Automatically adapt a supplied object to a required interface, and

\item Do all of the above, even when the components or frameworks involved
were not written to take advantage of this package, and even if the frameworks
have different mechanisms for defining interfaces.
\end{itemize}

Assuming that a particular framework either already supports this package, or
has been externally adapted to do so, then framework users will typically
use this package's declaration API to declare what interfaces their classes or
objects provide, and/or to declare adaptations between interfaces or
components.

For framework developers, this package offers an opportunity to replace
tedious and repetitive type-checking code (such as \function{isinstance()},
\function{type()}, \function{hasattr()}, or interface checks) with single
calls to \function{adapt()} instead.  In addition, if the framework has
objects that represent interfaces or protocols, the framework developer can
make them usable with this package's declaration API by adding adapters for
(or direct implementations of) the \class{IOpenProtocol} interface provided
herein.

If the developer of a framework does not do these things, it may still be
possible for a framework user or third-party developer to do them, in order to
be able to use this package's API.  The user of a framework can often call
\function{adapt()} on a component before passing it to a non-adapting
framework.  And, it's possible to externally adapt a
framework's interface objects as well.

For example, the \module{protocols.zope_support} and
\module{protocols.twisted_support} modules define adapters that
implement \class{IOpenProtocol} on behalf of Zope and Twisted \class{Interface}
objects.  This allows them to be used as arguments to this package's protocol
declaration API.  This works even though Zope and Twisted are completely unaware
of the \module{protocols} package.  (Of course, this does not give Zope or
Twisted \class{Interface} objects all of the capabilities that \class{Protocol}
objects have, but it does make most of their existing functionality accessible
through the same API.)

Finally, framework and non-framework developers alike may also wish to use the
\class{Protocol} and \class{Interface} base classes from this package to
define protocols or interfaces of their own, or perhaps use some of the
adaptation mechanisms supplied here to implement ``double dispatching'' or
the ``visitor pattern''.


\begin{seealso}

\seepep{246}{Object Adaptation}{PEP 246 describes an early version of the
adaptation protocol used by this package.}

\end{seealso}



































\subsection{Big Example 1 --- A Python Documentation Framework\label{protocols-example1}}

To provide the reader with a ``feel'' for the use and usefulness of the
\module{protocols} package, we will begin with a motivating example: a simple
Python documentation framework.  To avoid getting bogged down in details, we
will only sketch a skeleton of the framework, highlighting areas where the
\module{protocols} package would play a part.

First, let's consider the background and requirements.  Python has many
documentation tools available, ranging from the built-in \program{pydoc} to
third-party tools such as \program{HappyDoc}, \program{epydoc}, and
\program{Synopsis}.  Many of these tools generate documentation from ``live''
Python objects, some using the Python \module{inspect} module to do so.

However, such tools often encounter difficulties in the Python 2.2 world.
Tools that use \function{type()} checks break with custom metaclasses, and even
tools that use \function{isinstance()} break when dealing with custom
descriptor types.  These tools often handle other custom object types poorly as
well: for example, Zope \class{Interface} objects can cause some versions of
\function{help()} and \program{pydoc} to crash!

The state of Python documentation tools is an example of the problem that both
\pep{246} and the \module{protocols} package were intended to solve:
introspection makes frameworks brittle and unextensible.  We can't easily plug
new kinds of ``documentables'' into our documentation tools, or control how
existing objects are documented.  These are exactly the kind of problems that
component adaptation and open protocols were created to address.

So let's review our requirements for the documentation framework.  First, it
should work with existing built-in types, without requiring a new version of
Python.  Second, it should allow users to control how objects are recognized
and documented.  Third, we want the framework to be flexible enough to create
different kinds of documentation, like JavaDoc-style HTML, PDF reference
manuals, plaintext online help or manpages, and so on -- with whatever kinds of
documentable objects exist.  If user A creates a new ``documentable object,''
and user B creates a new documentation format, user C should be able to combine
the two.




To design a framework with the \module{protocols} package, the best place to
start is often with an ``ideal'' interface.  We pretend that every object is
already the kind of object that would do everything we need it to do.  In the
case of documentation, we want objects to be able to tell us their name, what
kind of object they should be listed as, their base classes (if applicable),
their call signature (if callable), and their contents (if a namespace).

So let's define our ideal interface, using \class{protocols.Interface}.
(\note{This is not the only way to define an interface; we could also use an
abstract base class, or other techniques.  Interface definition is
discussed further in section \ref{protocols-defining}.})

\begin{verbatim%
}from protocols import Interface

class IDocumentable(Interface):
    """A documentable object"""

    def getKind():
        """Return "kind" string, e.g. "class", "interface", "package", etc."""

    def getName():
        """Return an unqualified (undotted) name of the object"""

    def getBases():
        """Return a sequence of objects this object is derived from
        (or an empty sequence if object is not class/type-like)"""

    def getSignature():
        """Return a description of the object's call signature, or None"""

    def getSummary():
        """Return a one-line summary description of the object"""

    def getDoc():
        """Return documentation for the object (not including its contents)"""

    def getContents(includeInherited):
        """Return list of (key,value) pairs for namespace contents, if any"""
\end{verbatim}

Now, in the ``real world,'' no existing Python objects provide this interface.
But, if every Python object did, we'd be in great shape for writing
documentation tools.  The tools could focus purely on issues of formatting and
organization, not ``digging up dirt'' on what to say about the
objects.

Notice that we don't need to care whether an object is a type or a module or a
staticmethod or something else.  If a future version of Python made modules
callable or functions able to somehow inherit from other functions, we'd be
covered, as long as those new objects supported the methods we described.  Even
if a tool needs to care about the ``package'' kind vs. the ``module'' kind for
formatting or organizational reasons, it can easily be written to assume that
new ``kinds'' might still appear.  For example, an index generator might just
generate separate alphabetical indexes for each ``kind'' it encounters during
processing.

Okay, so we've envisioned our ideal scenario, and documented it as an
interface.  Now what?  Well, we could start writing documentation tools that
expect to be given objects that support the \class{IDocumentable} interface,
but that wouldn't be very useful, since we don't have any \class{IDocumentable}
objects yet.  So let's define some \strong{adapters} for built-in types, so
that we have something for our tools to document.



















\begin{verbatim%
}from protocols import advise
from types import FunctionType

class FunctionAsDocumentable(object):

    advise(
        # Declare that this class provides IDocumentable for FunctionType
        # (we'll explain this further in later sections)

        provides = [IDocumentable],
        asAdapterForTypes = [FunctionType],
    )

    def __init__(self, ob, protocol):
        self.func = ob

    def getKind(self):
        return "function"

    def getName(self):
        return self.func.func_name

    def getBases(self):
        return ()

    # ... etc.
\end{verbatim}

The \class{FunctionAsDocumentable} class wraps a function object with the
methods of an \class{IDocumentable}, giving us the behavior we need to document
it.  Now all we need to do is define similar wrappers for all the other
built-in types, and for any user-defined types, and then pick the right kind of
wrapper to use when given an object to document.

But wait! Isn't this just as complicated as writing a documentation tool the
``old fashioned'' way?  We still have to write code to get the data we need,
\emph{and} we still need to figure out what code to use.  Where's the benefit?



Enter the \pep{246} \function{adapt()} function.  \function{adapt()} has two
required arguments: an object to adapt, and a \strong{protocol} that you want
it adapted to.  Our documentation tool, when given an object to document, will
simply call \code{adapt(object,IDocumentable)} and receive an instance of the
appropriate adapter.  (Or, if the object has already declared that it supports
\class{IDocumentable}, we'll simply get the same object back from
\function{adapt()} that we passed in.)

But that's only the beginning.  If we create and distribute a documentation
tool based on \class{IDocumentable}, then anyone who creates a new kind of
documentable object can write their own adapters, and register them via the
\module{protocols} package, \emph{without} changing the documentation tool, or
needing to give special configuration options to the tool or call tool-specific
registration functions.  (Which means we don't have to design or code those
options or registration functions into our tool.)

Also, lets say I use some kind of ``finite state machine'' library written by
vendor A, and I'd like to use it with this new documentation tool from vendor
B.  I can write and register adapters from such types as ``finite state
machine,'' ``state,'' and ``transition'' to \class{IDocumentable}.  I can then
use vendor B's tool to document vendor A's object types.

And it goes further.  Suppose vendor C comes out with a new super-duper
documentation framework with advanced features.  To use the new features,
however, a more sophisticated interface than \class{IDocumentable} is needed.
So vendor C's tool requires objects to support his new
\class{ISuperDocumentable} interface.  What happens then?  Is the new package
doomed to sit idle because everybody else only has \class{IDocumentable}
objects?

Heck no.  At least, not if vendor C starts by defining an adapter from
\class{IDocumentable} to \class{ISuperDocumentable} that supplies reasonable
default behavior for ``older'' objects.  Then he or she writes adapters from
built-in types to \class{ISuperDocumentable} that provide more
useful-than-default behaviors where applicable.  Now, the super-framework is
instantly usable with existing adapters for other object types.  And, if vendor
C defined \class{ISuperDocumentable} as \emph{extending} \class{IDocumentable},
there's another benefit, too.



Suppose vendor A upgrades their finite state machine library to include direct
or adapted support for the new \class{ISuperDocumentable} interface.  Do I need
to switch to vendor C's documentation tool now?  Must I continue to maintain my
FSM-to-\class{IDocumentable} adapters?  No, to both questions.  If
\class{ISuperDocumentable} is a strict extension of \class{IDocumentable}, then
I may use an \class{ISuperDocumentable} anywhere an \class{IDocumentable} is
accepted.  Thus, I can use vendor A's new library with my existing
(non-``super'') documentation tool from vendor B, without needing my old
adapters any more.

As you can see, replacing introspection with adaptation makes frameworks more
flexible and extensible.  It also simplifies the design process, by letting you
focus on what you \emph{want} to do, instead of on the details of which objects
are doing it.

As we mentioned at the beginning of this section, this example is only a sketch
of a skeleton of a real documentation framework.  For example, a real
documentation framework would also need to define an \class{ISignature}
interface for objects returned from \method{getSignature()}.  We've also
glossed over many other issues that the designers of a real documentation
framework would face, in order to focus on the problems that can be readily
solved with adaptable components.

And that's our point, actually.  Every framework has two kinds of design
issues: the ones that are specific to the framework's subject area, and the
ones that would apply to any framework.  The \module{protocols} package can
save you a lot of work dealing with the latter, so you can spend more time
focusing on the former.  Let's start looking at how, beginning with the
concepts of ``protocols'' and ``interfaces''.












\subsection{Protocols and Interfaces \label{protocol-concepts}}

Many languages and systems provide ways of defining \strong{interfaces} that
components provide or require.  Some mechanisms are purely for documentation,
others are used at runtime to obtain or verify an implementation.  Typically,
interfaces are formal, intended for compiler-verified static type checking.

As a dynamic language, Python more often uses a looser notion of interface,
known as a \strong{protocol}.  While protocols are often very precisely
specified, their intended audience is a human reader or developer, not a
compiler or automated verification tool.

Automated verification tools, however, usually extract a high overhead cost
from developers.  The Java language, for example, requires that all methods
of an interface be defined by a class that claims to implement the
interface, even if those methods are never used in the program being
compiled!  And yet, the more important \emph{dynamic} behavior of the
interface at runtime is not captured or verifiable by the compiler, so written
documentation for human readers is still required!

In the Python language, the primary uses for objects representing protocols
or interfaces are at runtime, rather than at compile time.  Typically, such
objects are used to ask for an implementation of the interface, or supplied
by an object to claim that it provides an implementation of that interface.

In principle, any Python object may be used as a \strong{protocol object}.
However, for a variety of practical reasons, it is best that protocol objects
be hashable and comparable.  That is, protocol objects should be usable as
dictionary keys.

This still allows for a wide variety of protocol object implementations,
however.  One might assign meaning to the number 42, for example, as
referring to some hypothetical ``hitchhiker'' protocol.  More realistically,
the Microsoft COM framework uses UUIDs (Universally Unique Identifiers) to
identify interfaces.  UUIDs can be represented as Python strings, and thus
are usable as protocol objects.





But a simple string or number is often not very useful as a protocol
object.  Aside from the issue of how to assign strings or numbers to
protocols, these passive protocol objects cannot \emph{do} anything, and by
themselves they document nothing.

There are thus two more common approaches to creating protocol objects in
Python: classes (such as abstract base classes or ``ABCs''), and \strong{
interface objects}.  Interface objects are typically also defined using Python
\code{class} statements, but use a custom metaclass to create an object
that may not be usable in the same ways as a ``real'' Python class.  Many
Python frameworks (such as Twisted, Zope, and this package) provide their own
framework-specific implementations of this ``interface object'' approach.

Since classes and most interface object implementations can be used as
dictionary keys, and because their Python source code can serve as (or
be converted to) useful documentation, both of these approaches are viable
ways to create protocol objects usable with the \module{protocols} package.

In addition, inheriting from a class or interface objects is a simple way to
define implication relationships between protocol objects.  Inheriting from a
protocol to create a new protocol means that the new protocol \strong{implies}
the old protocol.  That is, any implementation or adaptation to the new
protocol, is implied to be usable in a place where the old protocol was
required.  (We will have more to say about direct and adapted implication
relationships later on, in section \ref{proto-implication}.)

At this point, we still haven't described any mechanisms for making adapters
available, or declaring what protocols are supported by a class or object.
To do that, we need to define two additional kinds of protocol objects, that
have more specialized abilities.

An \strong{adapting protocol} is a protocol object that is potentially able to
adapt components to support the protocol it represents, or at least to
recognize that a component supports (or claims to support) the protocol.  To
do this, an adapting protocol must have an \method{__adapt__} method, as
will be described in section \ref{adapt-protocol}.  (Often, this method
can be added to an existing class, or patched into an interface object
implementation.)



An \strong{open protocol} is an adapting protocol that is also capable of
accepting adapter declarations, and managing its implication relationships
with other protocols.  Open protocols can be used with this package's
protocol declaration API, as long as they implement (or can be adapted to)
the \class{IOpenProtocol} interface, as will be described in section
\ref{protocols-declaration-interfaces}.

Notice that the concepts of protocol objects, adapting protocols, and open
protocols are themselves ``protocols''.  The \module{protocols} package supplies
three interface objects that symbolize these concepts: \class{IProtocol},
\class{IAdaptingProtocol}, and \class{IOpenProtocol}, respectively.  Just as
the English phrases represent the concepts in this text, the interface objects
represent these concepts at runtime.

Whether a protocol object is as simple as a string, or as complex as an
\class{IOpenProtocol}, it can be used to request that a component provide
(or be adaptable to) the protocol that it symbolizes.  In the next section,
we'll look at how to make such a request, and how the different kinds of
protocol objects participate (or not) in fulfilling such requests.






















\subsection{\function{adapt()} and the Adaptation Protocol
\label{adapt-protocol}}

Component adaptation is the central focus of the \module{protocols} package.
All of the package's protocol declaration API depends on component adaptation
in order to function, and the rest of the package is just there to make it
easier for developers to use component adaptation in their frameworks and
programs.

Component adaptation is performed by calling the \function{adapt()} function,
whose design is based largely on the specification presented in \pep{246}:

\begin{funcdesc}{adapt}{component, protocol, \optional{, default}}

Return an implementation of \var{protocol} (a protocol object) for
\var{component} (any object).  The implementation returned may be
\var{component}, or a wrapper that implements the protocol on its
behalf.  If no implementation is available, return \var{default}.  If no
\var{default} is provided, raise \exception{protocols.AdaptationFailure}.

\end{funcdesc}

The component adaptation process performed by \function{adapt()} proceeds
in four steps:

\begin{enumerate}

\item If \var{protocol} is a class or type, and \var{component} is an instance
of that class or type, the component is returned unchanged.  (This quickly
disposes of the most trivial cases).

\item If \var{component} has a \method{__conform__} method, it is called,
passing in the protocol.  If the method returns a value other than
\constant{None}, it is returned as the result of \function{adapt()}.

\item If \var{protocol} has an \method{__adapt__} method, it is called,
passing in \var{component}.  If the method returns a value other than
\constant{None}, it is returned as the result of \function{adapt()}.



\item Perform default processing as described above, returning \var{default}
or raising \exception{protocols.AdaptationFailure} as
appropriate.

\end{enumerate}

This four-step process is called the \strong{adaptation protocol}.  Note
that it can be useful even in the case where neither the component nor the
protocol object are aware that the adaptation protocol exists, and it
gracefully degrades to a kind of \function{isinstance()} check in that
case.  However, if either the component or the protocol object has been
constructed (or altered) so that it has the appropriate \method{__conform__}
or \method{__adapt__} method, then much more meaningful results can be
achieved.

Throughout the rest of this document, we will say that a component
\strong{supports} a protocol, if calling \code{adapt(component,protocol)} does
not raise an error.  That is, a component supports a protocol if its
\method{__conform__} method or the protocol's \method{__adapt__} method
return a non-\constant{None} value.

This is different from saying that an object \strong{provides} a protocol.  An
object provides a protocol if \code{adapt(ob,protocol) is ob}.  Thus,
if an object \emph{provides} a protocol, it \emph{supports} the protocol, but
an object can also support a protocol by having an adapter that provides the
protocol on its behalf.

Now that you know how \function{adapt()} works, you can actually make use of it
without any of the other tools in the \module{protocols} package.  Just define
your own \method{__conform__} and \method{__adapt__} methods, and off
you go!










In practice, however, this is like creating a new kind of Python ``number''
type.  That is, it's certainly possible, but can be rather tedious and is
perhaps best left to a specialist.  For that reason, the \module{protocols}
package supplies some useful basic protocol types, and a ``declaration API''
that lets you declare how protocols, types, and objects should be adapted to
one another.  The rest of this document deals with how to use those types and
APIs.

You don't need to know about those types and APIs to create your own kinds of
protocols or components, just as you don't need to have studied Python's
numeric types or math libraries to create a numeric type of your own.  But,
if you'd like your new types to interoperate well with existing types, and
conform to users' expectations of how such a type behaves, it would be a good
idea to be familiar with existing implementations, such as the ones described
here.


























\subsubsection{Creating and Using Adapters, Components, and Protocols}

Because the adaptation protocol is so simple and flexible, there are a few
guidelines you should follow when using \function{adapt()} or creating
\method{__conform__} and \method{__adapt__} methods, to ensure that adapted
objects are as usable as unadapted objects.

First, adaptation should be \strong{reflexive}.  That is, if you
\function{adapt()} an object to a protocol, and then \function{adapt()} the
return value to the same protocol, the same object should be returned the
second time.  If you are using the \module{protocols} declaration API, it
suffices to declare that instances of the adapter class provide the protocol
they adapt to.  That is, if an adapter class provides protocol P for objects
of type X, then it should declare that it provides protocol P.

If you are not using the declaration API, but relying only upon your custom
\method{__conform__} and \method{__adapt__} methods, you need to ensure that
any adapters you return will return themselves when asked to support the
protocol that they were returned as an adapter for.

Second, adaptation is not automatically \strong{symmetric}.  That is, if I have
an object X that provides protocol P1, and I \function{adapt()} it to protocol
P2, it is not guaranteed that I can \function{adapt()} the resulting object to
P1 and receive the original object.  Ideally, someone who defines an adapter
function would also declare an inverse adapter function to ``unwrap'' an
adapted object to its original identity.  In practice, however, this can be
complex, since the adapter might need some fairly global knowledge of the
system to know when it is better to unwrap and rewrap, and when it is better to
further wrap the existing wrapper.

Another issue that occurs with such wrapper-based adaptation, is that the
wrapper does not have the same object identity as the base object, and may not
hash or compare equal to it, either.  Further, it is not guaranteed that
subsequent calls to \function{adapt()} will yield the same wrapper object -- in
fact it's quite unlikely.

These characteristics of adapted objects can be easily dealt with, however, by
following a few simple rules:



\begin{itemize}

\item Always \function{adapt()} from the ``original'' object you're supplied;
avoid adapting adaptations.

\item Always pass ``original'' objects to functions or methods that expect
their input to support more than one protocol; only pass adapted objects to
functions or methods that expect support for only one protocol.

\item Always use ``original'' objects for equality or identity comparisons --
or else ensure that callers know they will need to provide you with an equal or
identical adapter.  (One good way to document this, is to include the
requirement in the definition of the interface or protocol that your system
requires.)

\end{itemize}

In some respects, these rules are similar to dealing with objects in strongly
typed languages like Java.  In Java, if one simply has an ``object'', it is not
possible to perform operations specific to an interface, without first
``casting'' the object to that interface.  But, the object that was ``cast''
can't be stored in the same variable that the ``object'' was in, because it
is of a different type.

















\newpage
\subsubsection{Replacing Introspection with Adaptation\label{replintrowadapt}}

\begin{quotation}
To summarize: don't type check.

\hfill --- Alex Martelli, on \newsgroup{comp.lang.python}
\end{quotation}

Component adaptation is intended to completely replace all non-cooperative
introspection techniques, such as \function{type()}, \function{isinstance()},
\function{hasattr()}, and even interface checks.  Such introspection
tends to limit framework flexibility by unnecessarily closing policies to
extension by framework users.  It often makes code maintenance more difficult
as well, since such checks are often performed in more than one place, and
must be kept in sync whenever a new interface or type must be checked.

Some common use cases for such introspection are:

\begin{itemize}

\item To manually adapt a supplied component to a needed interface

\item To select one of several possible behaviors, based on the kind of
component supplied

\item To select another component, or take some action, using information
about the interfaces supported by the supplied component

\end{itemize}

Obviously, the first case is handled quite well by \function{adapt()}, at
least in an environment where it's easy to declare adapters between types and
protocols.  The second and third cases may at first seem to demand an ability
to introspect what interfaces are supported by a component.  But they are almost
always better served by defining new protocols that supply the required behavior
or metadata, and then requesting implementations of those protocols.





In all three use cases, replacing introspection with adaptation opens the
framework to third party extensions, without further modifications being
required -- and without the need to do extensive design or documentation
of a new hook or extension point to be added to the framework.  Indeed,
the availability of a standard mechanism for adaptation means that the
extension mechanism need only be documented once: right here in this
document.

In section \ref{introspect-elim}, we will present techniques for refactoring
all three kinds of introspection code to purely adaptation-driven
code, showing how the flexibility and readability of the code improves in the
process.  But first, we will need to cover how protocols and interfaces can
be defined, declared, and adapted, using the API provided by the
\module{protocols} package.

\begin{seealso}

\seetitle[http://www.canonical.org/\symbol{126}kragen/isinstance/]{isinstance()
Considered Harmful}{A brief critique of common justifications for using
introspection}

\end{seealso}


















\newpage
\subsubsection{Differences Between \function{protocols.adapt()} and \pep{246}}

If you have read \pep{246} or are looking for an exact implementation of it,
you should know that there are a few differences between the \module{protocols}
implementation of \function{adapt()} and the \pep{246} specification.  If you
don't care about these differences, you can skip this mini-appendix and
proceed directly to section \ref{protocols-defining}, ``Defining and Subclassing
Interfaces''.

The first difference is that \exception{TypeError} is treated differently in
each implementation.  \pep{246} says that if a \method{__conform__} or
\method{__adapt__} method raises a \exception{TypeError}, it should be
treated in the same way as if the method returned \constant{None}.  This was
a workaround for the issue of accidentally calling an unbound class
method, in the case where a component or protocol supplied to
\function{adapt()} was a class.

The \module{protocols} implementation of \function{adapt()} attempts to catch
such errors also, but will reraise any exception that appears to come from
\emph{within} the execution of the \method{__conform__} or
\method{__adapt__} method.  So if these methods raise a \exception{TypeError},
it will be passed through to the caller of \function{adapt}.  Thus, if you
are writing one of these methods, you should not raise a \exception{TypeError}
to signal the lack of an adaptation.  Rather, you should return \constant{None}.

Second, \exception{protocols.AdaptationFailure} is raised when no adaptation is
found, and no default is supplied, rather than the \exception{TypeError}
specified by \pep{246}.  (\note{\exception{protocols.AdaptationFailure} is a
subclass of \exception{TypeError} and \exception{NotImplementedError}, so code
written to catch either of these errors will work.})

These differences are the result of experience using the \module{protocols}
package with PEAK, and advances in the Python state-of-the-art since
\pep{246} was written (over two years ago).  We believe that they make the
adaptation protocol more robust, more predictable, and easier to use for
its most common applications.





\subsubsection{Convenience Adaptation API (NEW in 0.9.3)\label{protocols-calling}}

As of version 0.9.3, PyProtocols supports the simplified adaptation API that
was pioneered by Twisted, and later adopted by Zope.  In this simplified API,
a protocol can be called, passing in the object to be adapted.  So, for example,
instead of calling \code{adapt(foo,IBar)}, one may call \code{IBar(foo)}.  The
optional \var{default} and \var{factory} arguments may also be supplied,
following the \var{component} parameter.

All of the protocol types supplied by PyProtocols now support this simpler
calling scheme, except for \class{AbstractBase} subclasses, because calling
an \class{AbstractBase} subclass should create an instance of that subclass,
not attempt to adapt an arbitrary object.

Notice, by the way, that you should only use this simplified API if you know
for certain that the protocol supports it.  For example, it's safe to invoke
a known, constant interface object in this way.  But if you're writing code that
may receive a protocol object as a parameter or via another object, you should
use \function{adapt()} instead, because you may receive a protocol object that
does not support this shortcut API.




















\newpage
\subsection{Defining and Subclassing Interfaces \label{protocols-defining}}

The easiest way to define an interface with the \module{protocols} package is
to subclass \class{protocols.Interface}.  \class{Interface} does not supply any
data or methods of its own, so you are free to define whatever you need. There
are two common styles of defining interfaces, illustrated below:

\begin{verbatim%
}from protocols import Interface, AbstractBase

# "Pure" interface style

class IReadMapping(Interface):

    """A getitem-only mapping"""

    def __getitem__(key):
        """Return value for key"""


# Abstract Base Class (ABC) style

class AbstractMapping(AbstractBase):

    """A getitem-only mapping"""

    def __getitem__(self,key):
        """Return value for key"""
        raise NotImplementedError
\end{verbatim}

The ``pure'' style emphasizes the interface as seen by the caller, and is not
intended to be subclassed for implementation.  Notice that the \code{self}
parameter is not included in its method definitions, because \code{self} is not
supplied when calling the methods.  The ``ABC'' style, on the other hand,
emphasizes implementation, as it is intended to be subclassed
for that purpose.  Therefore, it includes method bodies, even for abstract
methods.  Each style has different uses: ``ABC'' is a popular rapid development
style, while the ``pure'' approach has some distinct documentation advantages.


\class{protocols.AbstractBase} may be used as a base class for either style, but
\class{protocols.Interface} is only usable for the "pure" interface style, as it
supports the convenience adaptation API (see section \ref{protocols-calling}).

(\note{both base classes use an explicit metaclass, so keep in mind that if
you want to subclass an abstract base for implementation using a different
metaclass, you may need to create a third metaclass that combines
\class{protocols.AbstractBaseMeta} with your desired metaclass.})

Subclassing a subclass of \class{Interface} (or \class{AbstractBase}) creates a
new interface (or ABC) that implies the first interface (or ABC).  This means
that any object that supports the second interface (or ABC), is considered to
implicitly support the first interface (or ABC).  For example:

\begin{verbatim%
}class IReadWriteMapping(IReadMapping):

    """Mapping with getitem and setitem only"""

    def __setitem__(key,value):
        """Store value for key, return None"""

\end{verbatim}

The \code{IReadWriteMapping} interface implies the \code{IReadMapping}
interface.  Therefore, any object that supports \code{IReadWriteMapping} is
understood to also support the \code{IReadMapping} interface.  The reverse,
however, is not true.

Inheritance is only one way to declare that one interface implies another,
however, and its uses are limited.  Let's say for example, that some package
\code{A} supplies objects that support \code{IReadWriteMapping}, while package
\code{B} needs objects that support \code{IReadMapping}.  But each package
declared its own interface, neither inheriting from the other.

As developers reading the documentation of these interfaces, it is obvious to
us that \code{IReadWriteMapping} implies \code{IReadMapping}, because we
understand what they do.  But there is no way for Python to know this, unless
we explicitly state it, like this:


\begin{verbatim%
}import protocols
from A import IReadWriteMapping
from B import IReadMapping

protocols.declareAdapter(
    protocols.NO_ADAPTER_NEEDED,
    provides = [IReadMapping],
    forProtocols = [IReadWriteMapping]
)
\end{verbatim}

In the above example, we use the \module{protocols} declaration API to say that
no adapter is needed to support the \code{B.IReadMapping} interface for
objects that already support the \code{A.IReadWriteMapping} interface.

At this point, if we supply an object that supports \code{IReadWriteMapping},
to a function that expects an \code{IReadMapping}, it should work, as long as
we call \code{adapt(ob,IReadMapping)} (or \code{IReadMapping(ob)}) first,
or the code we're calling does so.

There are still other ways to declare that one interface implies another.  For
example, if the author of our example package \code{B} knew about package {A}
and its \code{IReadWriteMapping} interface, he or she might have defined
\code{IReadMapping} this way:
















\begin{verbatim%
}import protocols
from protocols import Interface

from A import IReadWriteMapping

class IReadMapping(Interface):

    """A getitem-only mapping"""

    protocols.advise(
        protocolIsSubsetOf = [IReadWriteMapping]
    )

    def __getitem__(key):
        """Return value for key"""

\end{verbatim}

This is syntax sugar for creating the interface first, and then using
\code{protocols.declareAdapter(NO_ADAPTER_NEEDED)}.  Of course, you can only use
this approach if you are the author of the interface!  Otherwise, you must use
\function{declareAdapter()} after the fact, as in the previous example.

In later sections, we will begin looking at the \module{protocols} declaration
APIs -- like \function{declareAdapter()} and \function{advise()} -- in more
detail.  But first, we must look briefly at the interfaces that the
\module{protocols} package expects from the protocols, adapters, and other
objects supplied as parameters to the declaration API.











\newpage
\subsection{Interfaces Used by the Declaration API\label{protocols-declaration-interfaces}}

Like any other API, the \module{protocols} declaration API has certain
expectations regarding its parameters.  These expectations are documented and
referenced in code using interfaces defined in the \module{protocols.interfaces}
module.  (The interfaces are also exported directly from the top level of the
\module{protocols} package.)

You will rarely use or subclass any of these interface objects, unless you are
customizing or extending the system.  Four of the interfaces exist exclusively
for documentation purposes, while the rest are used in \function{adapt()} calls
made by the API.

First, let's look at the documentation-only interfaces.  It is not necessary
for you to declare that an object supports these interfaces, and the
\module{protocols} package never tries to \function{adapt()} objects to them.

\begin{description}

\item[IAdapterFactory] \hfill \\
Up until this point, we've been talking about ``adapters'' rather loosely.  The
\class{IAdapterFactory} interface formalizes the concept.  An \strong{adapter
factory} is a callable object that takes two arguments (an object and a
protocol, in that order) and returns an object that provides the protocol on
behalf of the passed-in object.  Declaration API functions that take ``adapter''
or ``factory'' arguments must be adapter factories.

The \module{protocols} package supplies two functions that provide
this interface: \function{NO_ADAPTER_NEEDED} and \function{DOES_NOT_SUPPORT}.
\function{NO_ADAPTER_NEEDED} is used to declare that an object provides a
protocol directly, and thus it returns the object passed into it, rather than
some kind of adapter.  \function{DOES_NOT_SUPPORT} is used to declare that an
object does not support a protocol, even with an adapter.  (Since this is the
default case, \function{DOES_NOT_SUPPORT} is rarely used, except to indicate
that a subclass does not support an interface that one of its superclasses
does.)





\item[IProtocol] \hfill \\
This interface formalizes the idea of a ``protocol object''.  As you will
recall from section \ref{protocol-concepts}, a protocol object is any object
that can be used as a Python dictionary key.  The second argument to
\function{adapt()} must be a protocol object according to this definition.


\item[IAdaptingProtocol] \hfill \\
This interface formalizes the idea of an ``adapting protocol'', specifically
that it is a protocol object (i.e., it provides \class{IProtocol}) that also
has an \method{__adapt__} method as described in section \ref{adapt-protocol}.
\class{IAdaptingProtocol} is a subclass of \class{IProtocol}, so of course
\function{adapt()} accepts such objects as protocols.

\item[IImplicationListener] \hfill \\
This interface is for objects that want to receive notification when new
implication relationships (i.e. adapters) are registered between two protocols.
If you have objects that want to keep track of what interfaces they support,
you may want those object to implement this interface so they can be kept
informed of new protocol-to-protocol adapters.

\end{description}

The other three interfaces are critical to the operation of the declaration API,
and thus must be supported by objects supplied to it.  The \module{protocols}
package supplies and registers various adapter classes that provide these
interfaces on behalf of many commonly used Python object types.  So, for each
interface, we will list ``known supporters'' of that interface, whether they
are classes supplied by \module{protocols}, or built-in types that are
automatically adapted to the interface.

We will not, however, go into details here about the methods and behavior
required by each interface.  (Those details can be found in section
\ref{protocols-interfaces-module}.)







\begin{description}

\item[IOpenProtocol] \hfill \\
This interface formalizes the ``open protocol'' concept that was introduced
in section \ref{protocol-concepts}.  An \class{IOpenProtocol} is an
\class{IAdaptingProtocol} that can also accept declarations made by the
\module{protocols} declaration API.

The \module{protocols} package supplies two implementations of this interface:
\class{Protocol} and \class{InterfaceClass}.  Thus, any \class{Interface}
subclass or \class{Protocol} instance is automatically considered to provide
\class{IOpenProtocol}.  \note{\class{Interface} is an instance of
\class{InterfaceClass}, and thus provides \class{IOpenProtocol}.  But if you
create an instance of an \class{Interface}, that object does not provide
\class{IOpenProtocol}, because the interfaces provided by an object and its
class (or its instances) can be different.}

In addition to its built-in implementations, the \module{protocols} package
also supplies and can declare adapter factories that adapt Zope X3 and Twisted's
interface objects to the \class{IOpenProtocol} interface, thus allowing
you to use Zope and Twisted interfaces in calls to the declaration API.  Similar
adapters for other frameworks' interfaces may be added, if there is sufficient
demand and/or contributed code, and the frameworks' authors do not add the
adapters to their frameworks.


\item[IOpenImplementor] \hfill \\
An \class{IOpenImplementor} is a class or type that can be told (via the
declaration API) what protocols its instances provide (or support via an
\class{IAdapterFactory}).  Note that this implies that the instances have
a \method{__conform__} method, or else they would not be able to tell
\function{adapt()} about the declared support!

Support for this interface is optional, since types that don't support it
can still have their instances be adapted by \class{IOpenProtocol} objects.
The \module{protocols} package does not supply any implementations or adapters
for this interface, either.  It is intended primarily as a hook for classes
to be able to receive notification about protocol declarations for their
instances.


\item[IOpenProvider] \hfill \\
Because objects' behavior usually comes from a class definition, it's not that
often that you will declare that a specific object provides or supports an
interface.  But objects like functions and modules do not have a class
definition, and classes themselves sometimes provide an interface.  (For
example, one could say that class objects provide an \class{IClass} interface.)
So, the declaration API needs to also be able to declare what protocols an
individual object (such as a function, module, or class) supports or provides.

That's what the \class{IOpenProvider} interface is for.  An
\class{IOpenProvider} is an object with a \class{__conform__} method, that can
be told (via the declaration API) what protocols it provides (or supports via
an \class{IAdapterFactory}).

Notice that this is different from \class{IOpenImplementor}, which deals with
an class or type's instances.  \class{IOpenProvider} deals with the object
itself.  A single object can potentially be both an \class{IOpenProvider} and an
\class{IOpenImplementor}, if it is a class or type.

The \module{protocols} package supplies and declares an adapter factory that
can adapt most Python objects to support this interface, assuming that they
have a \code{__dict__} attribute.  Thus, it is acceptable to pass a Python
function, module, or instance of a  ``classic'' class to any declaration API
that expects an \class{IOpenProvider} argument.

We'll talk more about making protocol declarations for individual objects
(as opposed to types) in section \ref{protocols-instances}, ``Protocol
Declarations for Individual Objects''.

\end{description}










\newpage
\subsection{Declaring Implementations and Adapters}
There are three kinds of relationships that a protocol can participate in:

\begin{itemize}
\item A relationship between a class or type, and a protocol its instances
provide or can be adapted to,

\item A relationship between an instance, and a protocol it provides or can
be adapted to, and

\item A relationship between a protocol, and another protocol that it implies
or can be adapted to.
\end{itemize}

Each of these relationships is defined by a \strong{source} (a type,
instance or protocol), a \strong{destination} (desired) protocol, and an
\strong{adapter factory} used to convert from one to the other.  If no adapter
is needed, we can say that the adapter factory is the special
\function{NO_ADAPTER_NEEDED} function.

To declare relationships like these, the \module{protocols} declaration API
provides three ``primitive'' declaration functions.  Each accepts a destination
protocol (that must support the \class{IOpenProtocol} interface),
an adapter factory (or \function{NO_ADAPTER_NEEDED}), and a source (type,
instance, or protocol).  These three functions are
\function{declareAdapterForType()}, \function{declareAdapterForObject()}, and
\function{declareAdapterForProtocol()}, respectively.

You will not ordinarily use these primitives, however, unless you are
customizing or extending the framework.  It is generally easier to call one
of the higher level functions in the declaration API.  These higher-level
functions may make several calls to the primitive functions on your behalf, or
supply useful defaults for certain parameters.  They are, however, based
entirely on the primitive functions, which is important for customizations and
extensions.

The next higher layer of declaration APIs are the explicit declaration
functions: \function{declareImplementation}, \function{declareAdapter}, and
\function{adviseObject}.  These functions are structured to support the most
common declaration use cases.

For declaring protocols related to a type or class:

\begin{funcdesc}{declareImplementation}{typ
\optional{, instancesProvide=[ ]} \optional{, instancesDoNotProvide=[ ]}}

Declare that instances of class or type \var{typ} do or do not provide
implementations of the specified protocols.  \var{instancesProvide} and
\var{instancesDoNotProvide} must be sequences of protocol objects that
provide (or are adaptable to) the \class{IOpenProtocol} interface,
such as \class{protocols.Interface} subclasses, or \class{Interface} objects
from Zope or Twisted.

This function is shorthand for calling \function{declareAdapterForType()}
with \function{NO_ADAPTER_NEEDED} and \function{DOES_NOT_SUPPORT} as adapters
from the type to each of the specified protocols.  Note, therefore, that the
listed protocols must be adaptable to \class{IOpenProtocol}.  See
\function{declareAdapterForType()} in section \ref{protocols-contents} for
details.
\end{funcdesc}


For declaring protocols related to a specific, individual instance:

\begin{funcdesc}{adviseObject}{ob
\optional{, provides=[ ]} \optional{, doesNotProvide=[ ]}}
Declare that \var{ob} provides (or does not provide) the specified protocols.
This is shorthand for calling \function{declareAdapterForObject()}
with \function{NO_ADAPTER_NEEDED} and \function{DOES_NOT_SUPPORT} as adapters
from the object to each of the specified protocols.  Note, therefore, that
\var{ob} may need to support \class{IOpenProvider}, and the listed protocols
must be adaptable to \class{IOpenProtocol}.  See
\function{declareAdapterForObject()} in section \ref{protocols-contents} for
details.  Also, see section \ref{protocols-instances}, ``Protocol Declarations
for Individual Objects'', for more information on using
\function{adviseObject}.
\end{funcdesc}





And for declaring all other kinds of protocol relationships:

\begin{funcdesc}{declareAdapter}{factory, provides,
\optional{, forTypes=[ ]} \optional{, forProtocols=[ ]}
\optional{, forObjects=[ ]}}

Declare that \var{factory} is an \class{IAdapterFactory} whose return value
provides the protocols listed in \var{provides} as an adapter for the
classes/types listed in \var{forTypes}, for objects providing the protocols
listed in \var{forProtocols}, and for the specific objects listed in
\var{forObjects}.

This function is shorthand for calling the primitive declaration
functions for each of the protocols listed in \var{provides} and each of the
sources listed in the respective keyword arguments.
\end{funcdesc}

Although these forms are easier to use than raw \code{declareAdapterForX}
calls, they still require explicit reference to the types or objects involved.
For the most common use cases, such as declaring protocol relationships to a
class, or declaring an adapter class, it is usually easier to use the ``magic''
\function{protocols.advise()} function, which we will discuss next.



















\subsubsection{Convenience Declarations in Class, Interface and Module Bodies \label{protcols-advise}}

Adapters, interfaces, and protocol implementations are usually defined in
Python \code{class} statements.  To make it more convenient to make protocol
declarations for these classes, the \module{protocols} package supplies the
\function{advise()} function.  This function  can make declarations about a
class, simply by being called from the body of that class.  It can also be
called from the body of a module, to make a declaration about the module.

\begin{funcdesc}{advise}{**kw}
Declare protocol relationships for the containing class or module.  All
parameters must be supplied as keyword arguments.  This function must be
called directly from a class or module body, or a \exception{SyntaxError}
results at runtime.  Different arguments are accepted, according to whether
the function is called within a class or module.

When invoked in the top-level code of a module, this function only accepts
the \code{moduleProvides} keyword argument.  When invoked in the body of a
class definition, this function accepts any keyword arguments \emph{except}
\code{moduleProvides}.  The complete list of keyword arguments follows.
Unless otherwise specified, protocols must support the \class{IOpenProtocol}
interface.

\note{When used in a class body, this function works by temporarily replacing
the \code{__metaclass__} of the class.  If your class sets an explicit
\code{__metaclass__}, it must do so \emph{before} \function{advise()} is
called, or the protocol declarations will not occur!}
\end{funcdesc}













Keyword arguments accepted by \function{advise()}:

\begin{description}

\item[instancesProvide = \var{protocols}] \hfill \\
A sequence of protocols that instances of the containing class provide, without
needing an adapter.  Supplying this argument is equivalent to calling
\code{declareImplementation(\var{containing class},\var{protocols})}.

\item[instancesDoNotProvide = \var{protocols}] \hfill \\
A sequence of protocols that instances of the containing class do not provide.
This is primarily intended for ``rejecting'' protocols provided or supported
by base classes of the containing class.  Supplying this argument is equivalent
to calling \code{declareImplementation(\var{containing
class},instancesDoNotProvide=\var{protocols})}.

\item[asAdapterForTypes = \var{types}] \hfill \\
Declare the containing class as an adapter for \var{types}, to the protocols
listed by the \code{instancesProvide} argument (which must also be supplied).
Supplying this argument is equivalent to calling
\code{declareAdapter(\var{containing class}, \var{instancesProvide},
forTypes=\var{types})}.  (Note that this means the containing class must be an
object that provides \class{IAdapterFactory}; i.e., its constructor should
accept being called with two arguments: the object to be adapted, and the
target protocol.)

\item[asAdapterForProtocols = \var{protocols}] \hfill \\
Declare the containing class as an adapter for \var{protocols}, to the
protocols listed by the \code{instancesProvide} argument (which must also be
supplied).  Supplying this argument is equivalent to calling
\code{declareAdapter(\var{containing class}, \var{instancesProvide},
forProtocols=\var{types})}.  (Note that this means the containing class must be
an object that provides \class{IAdapterFactory}; i.e., its constructor should
accept being called with two arguments: the object to be adapted, and the
target protocol.)

\item[factoryMethod = \var{methodName}] \hfill \\ \versionadded{0.9.1}
When using \code{asAdapterForTypes} or \code{asAdapterForProtocols}, you can
also supply a factory method name, using this keyword.  The method named must be
a \emph{class} method, and it will be used in place the class' normal constructor.

\item[protocolExtends = \var{protocols}] \hfill \\
Declare that the containing class is a protocol that extends (i.e., implies)
the listed protocols.  This keyword argument is intended for use inside class
statements that themselves define protocols, such as \class{Interface}
subclasses, and that need to ``inherit'' from incompatible protocols.  For
example, an \class{Interface} cannot directly subclass a Zope interface,
because their metaclasses are incompatible.  But using \code{protocolExtends}
works around this:

\begin{verbatim%
}import protocols
from mypackage import ISomeInterface
from zope.something.interfaces import ISomeZopeInterface

class IAnotherInterface(ISomeInterface):
    protocols.advise(
        protocolExtends = [ISomeZopeInterface]
    )
    #... etc.
\end{verbatim}

In the above example, \code{IAnotherInterface} wants to extend both
\code{ISomeInterface} and \code{ISomeZopeInterface}, but cannot do so directly
because the interfaces are of incompatible types.  \code{protocolExtends}
informs the newly created interface that it implies \code{ISomeZopeInterface},
even though it isn't derived from it.

Using this keyword argument is equivalent to calling
\code{declareAdapter(NO_ADAPTER_NEEDED, \var{protocols},
forProtocols=[\var{containing class}])}.  Note that this means that the
containing class must be an object that supports \class{IOpenProtocol}, such
as an \class{Interface} subclass.









\item[protocolIsSubsetOf = \var{protocols}] \hfill \\
Declare that the containing class is a protocol that is implied (extended) by
the listed protocols.  This is just like \code{protocolExtends}, but in the
``opposite direction''.  It allows you to declare (in effect) that some other
interface is actually a subclass of (extends, implies) this one.  See the
examples in section \ref{protocols-defining} for illustration.

Using this keyword argument is equivalent to calling
\code{declareAdapter(NO_ADAPTER_NEEDED, [\var{containing class}],
forProtocols=\var{protocols})}.

\item[equivalentProtocols = \var{protocols}] \hfill \\
\versionadded{0.9.1}
Declare that the containing class is a protocol that is equivalent to the
listed protocols.  That is, the containing protocol both implies, and is implied
by, the listed protocols.  This is a convenience feature intended mainly to
support the use of generated protocols.

Using this keyword is equivalent to using both the \code{protocolExtends} and
\code{protocolIsSubsetOf} keywords, with the supplied protocols.

\item[classProvides = \var{protocols}] \hfill \\
Declare that the containing class \emph{itself} provides the specified
protocols.  Supplying this argument is equivalent to calling
\code{adviseObject(\var{containing class}, \var{protocols})}.  Note that this
means that the containing class may need to support the \class{IOpenProvider}
interface.  The \module{protocols} package supplies default adapters to
support \class{IOpenProvider} for both classic and new-style classes, as long
as they do not have custom \method{__conform__} methods.  See section
\ref{protocols-instances}, ``Protocol Declarations for Individual Objects''
for more details.

\item[classDoesNotProvide = \var{protocols}] \hfill \\
Declare that the containing class \emph{itself} does not provide the specified
protocols.  This is for classes that need to reject inherited class-level
\code{classProvides} declarations.  Supplying this argument is equivalent to
calling \code{adviseObject(\var{containing class},
doesNotProvide=\var{protocols})}, and the \class{IOpenProvider} requirements
mentioned above for \code{classProvides} apply here as well.


\item[moduleProvides = \var{protocols}] (module context only) \hfill \\
A sequence of protocols that the enclosing module provides.  Equivalent to
\code{adviseObject(\var{containing module}, \var{protocols})}.

\end{description}



































\newpage
\subsubsection{Protocol Declarations for Individual Objects \label{protocols-instances}}

Because objects' behavior usually comes from a class definition, it's not too
often that you will declare that an individual object provides or supports an
interface, as opposed to making a blanket declaration about an entire class or
type of object.  But objects like functions and modules do not \emph{have} a
class definition that encompasses their behavior, and classes themselves
sometimes provide an interface (e.g. via \function{classmethod} objects).

So, the declaration API needs to also be able to declare what protocols an
individual object (such as a function, module, or class) supports or provides.
This is what \function{adviseObject()} and the
\code{classProvides}/\code{classDoesNotProvide} keywords of
\function{advise()} do.

In most cases, for an object to be usable with \function{adviseObject()}, it
must support the \class{IOpenProvider} interface.  Since many of the objects
one might wish to use with \function{adviseObject()} (such as modules,
functions, and classes) do not directly provide this interface, the
\module{protocols.classic} module supplies and declares an adapter factory tha
can adapt most Python objects to support this interface, assuming that they
have a \code{__dict__} attribute.

This default adapter works well for many situations, but it has some
limitations you may need to be aware of.  First, it works by ``poking'' a new
\method{__conform__} method into the adapted object.  If the object already
has a \method{__conform__} method, a \exception{TypeError} will be raised.  So,
if you need an object to be an \class{IOpenProvider}, but it has a
\method{__conform__} method, you may want to have its class include
\class{ProviderMixin} among its base classes, so that your objects won't rely
on the default adapter for \class{IOpenProvider}.  (See \class{ProviderMixin}
in section \ref{protocols-contents} for more on this.)

Both the default adapter and \class{ProviderMixin} support inheritance of
protocol declarations, when the object being adapted is a class or type.  In
this way, \code{advise(classProvides=\var{protocols})} declarations (or
\code{adviseObject(someClass,\var{protocols})} calls) are inherited by
subclasses.  Of course, you can always reject inherited protocol information
using \code{advise(classDoesNotProvide=\var{protocols})} or
\function{adviseObject(newClass,doesNotProvide=\var{protocols})}.

Both the default adapter and \class{ProviderMixin} work by keeping a mapping of
protocols to adapter factories.  Keep in mind that this means the protocols and
adapter factories will continue to live until your object is garbage collected.
Also, that means for your object to be pickleable, all of the protocols and
adapter factories used must be pickleable.  (This latter requirement can be
quite difficult to meet, since composed adapter factories are dynamically
created functions at present.)

Note that none of these restrictions apply if you are only using declarations
about types and protocols, as opposed to individual objects.  (Or if you only
make individual-object declarations for functions, modules, and classes.)
Also note that if you have some objects that need to dynamically support or
not support a protocol on a per-instance basis, then \function{adviseObject()}
is probably not what you want anyway!  Instead, give your objects' class a
\method{__conform__()} method that does the right thing when the object is
asked to conform to a protocol.  \function{adviseObject()} is really intended
for adding metadata to objects that ``don't know any better''.

In general, protocol declarations are a \emph{static} mechanism: they cannot be
changed or removed at will, only successively refined.  All protocol
declarations made must be consistent with the declarations that have already
been made.  This makes them unsuitable as a mechanism for dynamic behavior such
as supporting a protocol based on an object's current state.

In the next section, we'll look more at the static nature of declarations,
and explore what it means to make conflicting (or refining) protocol
declarations.













\newpage
\subsection{Protocol Implication and Adapter Precedence \label{proto-implication}}

So far, we've only dealt with simple one-to-one relationships between
protocols, types, and adapter factories.  We haven't looked, for example, at
what happens when you define that class X instances provide interface IX,
that AXY is an adapter factory that adapts interface IX to interface IY, and
class Z subclasses class X.  (As you might expect, what happens is that Z
instances will be wrapped with an AXY adapter when you call
\code{adapt(instanceOfZ, IY)}.)

Adaptation relationships declared via the declaration API are
\strong{transitive}.  This means that if you declare an adaptation from item A
to item B, and from item B to item C, then there is an \strong{adapter path}
from A to C.  An adapter path is effectively a sequence of adapter factories
that can be applied one by one to get from a source (type, object, or protocol)
to a desired destination protocol.

Adapter paths are automatically composed by the types, objects, and protocols
used with the declaration API, using the \function{composeAdapters()} function.
Adapter paths are said to have a \strong{depth}, which is the number of steps
taken to get from the source to the destination protocol.  For example, if
factory AB adapts from A to B, and factory BC adapts from B to C, then an
adapter factory composed of AB and BC would have a depth of 2.  However, if we
registered another adapter, AC, that adapts directly from A to C, this adapter
path would have a depth of 1.

Naturally, adapter paths with lesser depth are more desirable, as they are less
likely to be a ``lossy'' conversion, and are more likely to be efficient.  For
this reason, shorter paths take precedence over longer paths.  Whenever an
adapter factory is declared between two points that previously required a
longer path, all adapter paths that previously included the longer path segment
are updated to use the newly shortened route.  Whenever an adapter factory is
declared that would \emph{lengthen} an existing path, it is ignored.

The net result is that the overall network of adapter paths will tend to
stabilize over time.  As an added benefit, it is safe to define circular
adapter paths (e.g. A to B, B to C, C to A), as only the shortest useful
adapter paths are generated.



We've previously mentioned the special adapter factories
\function{NO_ADAPTER_NEEDED} and \function{DOES_NOT_SUPPORT}.  There are a
couple of special rules regarding these adapters that we need to add.  Any
adapter path that contains \function{DOES_NOT_SUPPORT} can be reduced to a
single instance of \function{DOES_NOT_SUPPORT}, and any adapter path that
contains \function{NO_ADAPTER_NEEDED} is equivalent to the same adapter path
without it. These changes can be used to simplify adapter paths, but are only
taken into consideration when comparing paths, if the ``unsimplified'' version
of the adapter paths are the same length.

Lets' consider two adapter paths between A and C.  Each proceeds by way of B.
(i.e., they go from A to B to C.)  Which one is preferable?  Both
adapters have a depth of 2, because there are two steps (A to B, B to C).  But
suppose one adapter path contains two arbitrary adapter factories, and the
other is composed of one factory plus \function{NO_ADAPTER_NEEDED}.  Clearly,
that path is superior, since it effectively contains only one adapter instead
of two.

This simplification, however, can \emph{only} be applied when the unsimplified
paths are of the same length.  Why?  Consider our example of two paths from A
to B to C.  If someone declares a direct path from A to C (i.e. not via B or
any other intermediate protocol), we want this path to take precedence over an
indirect path, even if both paths ``simplify'' to the same length.  Only if we
are choosing between two paths with the same number of steps can we can use the
length of their simplified forms as a ``tiebreaker''.

So what happens when choosing between paths of the same number of steps and the
same simplified length?  A \exception{TypeError} occurs, unless one of these
conditions applies:

\begin{itemize}
\item One of the paths simplifies to \function{DOES_NOT_SUPPORT}, in which case
the other path is considered preferable.  (Some ability is better than none.)

\item One of the paths simplifies to \function{NO_ADAPTER_NEEDED}, in which
case it is considered preferable.  (It's better not to have to adapt.)

\item Both of the paths are the same object, in which case no change is
required to the existing path.  (The declaration is redundant.)
\end{itemize}

Notice that this means that it is not possible to override an existing adapter
path unless you are improving on it a way visible to the system.  This doesn't
mean, however, that you can't take advantage of existing declarations, while
still overriding some of them.

Suppose that there exists a set of existing adapters and protocols defined by
some frameworks, and we are writing an application using them.  We would like,
however, for our application to be able to override certain existing
relationships.  Say for example that we'd like to have an adapter path from A
to C that's custom for our application, but we'd like to ``inherit'' all the
other adaptations to C, so that by default any C implementation is still useful
for our application.

The simple solution is to define a new protocol D as a \strong{subset} of
protocol C.  This is effectively saying that \function{NO_ADAPTER_NEEDED} can
adapt from C to D. All existing declarations adapting to C, are now usable as
adaptations to D, but they will have lower precedence than any direct
adaptation to D. So now we define our direct adaptation from A to D, and it
will take precedence over any A to C to D path. But, any existing path that
goes to C will be ``inherited'' by D.

Speaking of inheritance, please note that inheritance between types/classes has
no effect on adapter path depth calculations.  Instead, any path defined for a
subclass takes absolute precedence over paths defined for a superclass, because
the subclass is effectively a different starting point.  In other words, if A
is a class, and Q subclasses A, then an adapter path between Q and some
protocol is a different path than the path between A and that protocol.  There
is no comparison between the two, and no conflict.  However, if a path from Q
to a desired protocol does not exist, then the existing best path for A will be
used.

Sometimes, one wishes to subclass a class without taking on its full
responsibilities.  It may be that we want Q to use A's implementation, but we
do not want to support some of A's protocols. In that case, we can declare
\function{DOES_NOT_SUPPORT} adapters for those protocols, and these will ensure
that the corresponding adapter paths for A are not used.





This is called \strong{rejecting inherited declarations}.  It is not,
generally speaking, a good idea.  If you want to use an existing class'
implementation, but do not wish to abide by its contracts (protocols), you
should be using \strong{delegation} rather than inheritance.  That is, you
should define your new class so that it has an attribute that is an instance of
the old class.  For example, if you are tempted to subclass Python's built-in
dictionary type, but you do not want your subclass to really \emph{be} a
dictionary, you should simply have an attribute that is a dictionary.

Because rejecting inherited declarations is a good indication that inheritance
is being used improperly, the \module{protocols} package does not encourage the
practice.  Declaring a protocol as \function{DOES_NOT_SUPPORT} does not
propagate to implied protocols, so every rejected protocol \emph{must} be
listed explicitly.  If class A provided protocol B, and protocol B derived
from (i.e. implied) protocol C, then you must explicitly reject both B and C
if you do not want your subclass to support them.

\begin{seealso}

The logic of composing and comparing adapter paths is implemented via the
\function{composeAdapters()} and \function{minimumAdapter()} functions in the
\module{protocols.adapters} module.  See section \ref{protocol-adapters-module}
for more details on these and other functions that relate to adapter paths.


\end{seealso}














\newpage
\subsection{Dynamic Protocols (NEW in 0.9.1)\label{protocols-generated}}

For many common uses of protocols, it may be inconvenient to subclass
\class{protocols.Interface} or to manually create a \class{Protocol} instance.
So, the \module{protocols} package includes a number of utility functions to
make these uses more convenient.



































\subsubsection{Defining a protocol based on a URI or UUID}\label{protocols-generated-uri}
\begin{funcdesc}{protocolForURI}{uri}\versionadded{0.9.1}
Return a protocol object that represents the supplied URI or UUID string.  It
is guaranteed that you will receive the same protocol object if you call this
routine more than once with equal strings.  This behavior is preserved even
across pickling and unpickling of the returned protocol object.

The purpose of this function is to permit modules to refer to protocols defined
in another module, that may or may not be present at runtime.  To do this, a
protocol author can declare that their protocol is equivalent to a URI string:

\begin{verbatim%
}from protocols import advise, Interface, protocolForURI

class ISomething(Interface):
    advise(
        equivalentProtocols = [protocolForURI("some URI string")]
    )
    # etc...
\end{verbatim}

Then, if someone wishes to use this protocol without importing
\code{ISomething} (and thereby becoming dependent on the module that provides
it), they can do something like:

\begin{verbatim%
}from protocols import advise, protocolForURI

class MyClass:
    advise(
        provides = [protocolForURI("some URI string")]
    )
    # etc...
\end{verbatim}

Thus, instances of \code{MyClass} will be considered to support
\code{ISomething}, if needed.  But, if \code{ISomething} doesn't exist, no
error occurs.
\end{funcdesc}


\subsubsection{Defining a protocol as a subset of an existing type}\label{protocols-generated-type}
\begin{funcdesc}{protocolForType}{baseType,
\optional{methods=(), implicit=False}}
Return a protocol object that represents the subset of \var{baseType} denoted
by \var{methods}.  It is guaranteed that you will receive the same protocol
object if you call this routine more than once with eqivalent paremeters.  This
behavior is preserved even across pickling and unpickling of the returned
protocol object.

\var{baseType} should be a type object, and \var{methods} should be a sequence
of attribute or method names.  (The order of the names is not important.)  The
\var{implicit} flag allows adapting objects that don't explicitly declare
support for the protocol.  (More on this later.)

Typical usage of this function is to quickly define a simple protocol based on
a Python built-in type such as \class{list}, \class{dict}, or \class{file}:

\begin{verbatim%
}IReadFile = protocols.protocolForType(file, ['read','close'])
IReadMapping = protocols.protocolForType(dict, ['__getitem__'])
\end{verbatim}

The advantage of using this function instead of creating an \class{Interface}
subclass is that users do not need to import your specific \class{Interface}
definition.  As long as they declare support for a protocol based on the
same type, and with at least the required methods, then their object will
be considered to support the protocol.  For example, declaring that you support
\code{protocolForType(file, ['read','write','close'])} automatically implies
that you support \code{protocolForType(file, ['read','close'])} and
\code{protocolForType(file, ['write','close'])} as well.  (Note: instances of
the \var{baseType} and its subclasses will also be considered to provide the
returned protocol, whether or not they explicitly declare support.)

If you supply a true value for the \var{implicit} flag, the returned protocol
will also adapt objects that have the specified methods or attributes.  In
other words, \code{protocolForType(file, ['read','close'], True)} returns a
protocol that will consider any object with \method{read} and \method{close}
methods to provide that protocol, as well as objects that explicitly support
\code{protocolForType(file, ['read','close'])}.


In order to automatically declare the relationships between the protocols for
different subsets, this function internally generates all possible subsets of
a requested \var{methods} list.  So, for example, requesting a protocol with
8 method names may cause as many as 127 protocol objects to be created.  Of
course, these are generated only once in the lifetime of the program, but you
should be aware of this if you are using large method subsets.  Using as few as
32 method names would create 2 billion protocols!

Note also that the supplied \var{baseType} is used only as a basis for
semantic distinctions between sets of similar method names, and to declare that
the \var{baseType} and its subclasses support the returned protocol.  No
protocol-to-protocol relationships are automatically defined between protocols
requested for different base types, regardless of any subclass/superclass
relationship between the base types.
\end{funcdesc}

\subsubsection{Defining a protocol for a sequence}\label{protocols-generated-sequence}
\begin{funcdesc}{sequenceOf}{protocol} \versionadded{0.9.1}
Return a protocol object that represents a sequence of objects adapted to
\var{protocol}.  Thus, \code{protocols.sequenceOf(IFoo)} is a protocol that
represents a \class{protocols.IBasicSequence} of objects supporting the
\class{IFoo} protocol.  It is guaranteed that you will receive the same protocol
object if you call this routine more than once with the same protocol, even
across pickling and unpickling of the returned protocol object.

When this function creates a new sequence protocol, it automatically declares
an adapter function from \class{protocols.IBasicSequence} to the new protocol.
The adapter function returns the equivalent of \code{[adapt(x,protocol) for x
in sequence]}, unless one of the adaptations fails, in which case it returns
\code{None}, causing the adaptation to fail.

The built-in \class{list} and \class{tuple} types are declared as
implementations of \class{protocols.IBasicSequence}, so protocols returned by
\function{sequenceOf()} can be used immediately to convert lists or tuples into
lists of objects supporting \var{protocol}.  If you need to adapt other kinds
of sequences using your \function{sequenceProtocol()}, you will need to declare
that those sequences implement \class{protocols.IBasicSequence} unless they
subclass \class{tuple}, \class{list}, or some other type that implements
\class{protocols.IBasicSequence}.
\end{funcdesc}

\subsubsection{Defining a protocol as a local variation of another protocol}\label{protocols-generated-local}

\begin{classdesc}{Variation}{baseProtocol \optional{, context=None}}
\versionadded{0.9.1}
A \class{Variation} is a \class{Protocol} that "inherits" adapter declarations
from an existing protocol.  When you create a \class{Variation}, it declares
that it is implied by its \var{baseProtocol}, and so any adpater suitable for
adapting to the base protocol is therefore suitable for the \class{Variation}.
This allows you to then declare adapters to the variation protocol, without
affecting those declared for the base protocol.  In this way, you can have a
protocol object that represents the use of the base protocol in a particular
context.  You can optionally specify that context via the \var{context}
argument, which will then serve as the \member{context} attribute of the
protocol.  For more background on how this works and what it might be used for,
see section \ref{protocols-context}.
\end{classdesc}
























\newpage
\subsection{Package Contents and Contained Modules\label{protocols-contents}}

The following functions, classes, and interfaces are available from the
top-level \module{protocols} package.

\begin{funcdesc}{adapt}{component, protocol \optional{, default}}

Return an implementation of \var{protocol} (a protocol object) for
\var{component} (any object).  The implementation returned may be
\var{component}, or an adapter that implements the protocol on its
behalf.  If no implementation is available, return \var{default}.  If no
\var{default} is provided, raise \exception{protocols.AdaptationFailure}.

A detailed description of this function's operations and purpose may be found in
section \ref{adapt-protocol}.
\end{funcdesc}

\begin{excdesc}{AdaptationFailure}
\versionadded{0.9.3}
A subclass of \exception{TypeError} and \exception{NotImplementedError}, this
exception type is raised by \function{adapt()} when no implementation can be
found, and no \var{default} was supplied.
\end{excdesc}

\begin{classdesc}{Adapter}{ob, proto}
\versionadded{0.9.1}
This base class provides a convenient \method{__init__} method for adapter
classes.  To use it, just subclass \class{protocols.Adapter} and add methods
to implement the desired interface(s).  (And of course, declare what interfaces
the adapter provides, for what types, and so on.)  Your subclass' methods can
use the following attributes, which will have been set by the \method{__init__}
method:









\begin{memberdesc}{subject}
The \member{subject} attribute of an \class{Adapter} instance is the \var{ob}
supplied to its constructor.  That is, it is the object being adapted.
\end{memberdesc}

\begin{memberdesc}{protocol} The \member{subject} attribute of an
\class{Adapter} instance is the \var{proto} supplied to its constructor.  That
is, it is the protocol to which the \member{subject} is being adapted.
\end{memberdesc}
\end{classdesc}

\begin{funcdesc}{advise}{**kw}
Declare protocol relationships for the containing class or module.  All
parameters must be supplied as keyword arguments.  This function must be
called directly from a class or module body, or a \exception{SyntaxError}
results at runtime.  Different arguments are accepted, according to whether
the function is called within a class or module.

When invoked in the top-level code of a module, this function only accepts
the \code{moduleProvides} keyword argument.  When invoked in the body of a
class definition, this function accepts any keyword arguments \emph{except}
\code{moduleProvides}.  The complete list of keyword arguments can be found
in section \ref{protcols-advise}.

\note{When used in a class body, this function works by temporarily replacing
the \code{__metaclass__} of the class.  If your class sets an explicit
\code{__metaclass__}, it must do so \emph{before} \function{advise()} is
called, or the protocol declarations will not occur!}
\end{funcdesc}

\begin{funcdesc}{adviseObject}{ob \optional{, provides=[ ]} \optional{, doesNotProvide=[ ]}}
Declare that \var{ob} provides (or does not provide) the specified protocols.
This is shorthand for calling \function{declareAdapterForObject()}
with \function{NO_ADAPTER_NEEDED} and \function{DOES_NOT_SUPPORT} as adapters
from the object to each of the specified protocols.  Note, therefore, that
\var{ob} may need to support \class{IOpenProvider}, and the listed protocols
must be adaptable to \class{IOpenProtocol}.  See section
\ref{protocols-instances}, ``Protocol Declarations for Individual Objects'',
for more information on using \function{adviseObject}.
\end{funcdesc}

\begin{classdesc}{Attribute}{doc\optional{,name=\constant{None}, value=\constant{None}}}
This class is used to document attributes required by an interface.  An example
usage:

\begin{verbatim%
}from protocols import Interface, Attribute

class IFoo(Interface):

    Bar = Attribute("""All IFoos must have a Bar attribute""")
\end{verbatim}

If you are using the ``Abstract Base Class'' or ABC style of interface
documentation, you may wish to also use the \var{name} and \var{value}
attributes.  If supplied, the \class{Attribute} object will act as a data
descriptor, supplying \var{value} as a default value, and storing any newly set
value in the object's instance dictionary.  This is useful if you will be
subclassing the abstract base and creating instances of it, but still want to
have documentation appear in the interface.  When the interface is displayed
with tools like \program{pydoc} or \function{help()}, the attribute
documentation will be shown.
\end{classdesc}


\begin{funcdesc}{declareAdapter}{factory, provides
\optional{, forTypes=[ ]} \optional{, forProtocols=[ ]}
\optional{, forObjects=[ ]}}

Declare that \var{factory} is an \class{IAdapterFactory} whose return value
provides the protocols listed in \var{provides} as an adapter for the
classes/types listed in \var{forTypes}, for objects providing the protocols
listed in \var{forProtocols}, and for the specific objects listed in
\var{forObjects}.

This function is shorthand for calling the primitive declaration
functions (\function{declareAdapterForType},
\function{declareAdapterForProtocol}, and \function{declareAdapterForObject})
for each of the protocols listed in \var{provides} and each of the
items listed in the respective keyword arguments.
\end{funcdesc}

\begin{funcdesc}{declareImplementation}{typ
\optional{, instancesProvide=[ ]} \optional{, instancesDoNotProvide=[ ]}}

Declare that instances of class or type \var{typ} do or do not provide
implementations of the specified protocols.  \var{instancesProvide} and
\var{instancesDoNotProvide} must be sequences of protocol objects that
provide (or are adaptable to) the \class{IOpenProtocol} interface,
such as \class{protocols.Interface} subclasses, or \class{Interface} objects
from Zope or Twisted.

This function is shorthand for calling \function{declareAdapterForType()}
with \function{NO_ADAPTER_NEEDED} and \function{DOES_NOT_SUPPORT} as adapters
from the type to each of the specified protocols.  Note, therefore, that the
listed protocols must be adaptable to \class{IOpenProtocol}.
\end{funcdesc}


\begin{funcdesc}{DOES_NOT_SUPPORT}{component, protocol}
This function simply returns \constant{None}.  It is a placeholder used whenever
an object, type, or protocol does not implement or imply another protocol.
Whenever adaptation is not possible, but the \module{protocols} API function
you are calling requires an adapter, you should supply this function as the
adapter.  Some protocol implementations, such as the one for Zope interfaces,
are unable to handle adapters other than \function{NO_ADAPTER_NEEDED} and
\function{DOES_NOT_SUPPORT}.
\end{funcdesc}


\begin{classdesc*}{Interface}
Subclass this to create a "pure" interface.  See section \ref{protocols-defining}
for more details.
\end{classdesc*}

\begin{classdesc*}{AbstractBase}
\versionadded{0.9.3}
Subclass this to create an "abstract base class" or "ABC" interface.  See
section \ref{protocols-defining} for more details.
\end{classdesc*}



\begin{funcdesc}{NO_ADAPTER_NEEDED}{component, protocol}
This function simply returns \var{component}.  It is a placeholder used whenever
an object, type, or protocol directly implements or implies another protocol.
Whenever an adapter is not required, but the \module{protocols} API function
you are calling requires an adapter, you should supply this function as the
adapter.  Some protocol implementations may be unable to handle adapters other
than \function{NO_ADAPTER_NEEDED} and \function{DOES_NOT_SUPPORT}.
\end{funcdesc}

\begin{funcdesc}{protocolForType}{baseType,
\optional{methods=(), implicit=False}} \versionadded{0.9.1}
Return a protocol object that represents the subset of \var{baseType} denoted
by \var{methods}.  It is guaranteed that you will receive the same protocol
object if you call this routine more than once with eqivalent paremeters.  This
behavior is preserved even across pickling and unpickling of the returned
protocol object.

\var{baseType} should be a type object, and \var{methods} should be a sequence
of attribute or method names.  (The order of the names is not important.)  The
\var{implicit} flag allows adapting objects that don't explicitly declare
support for the protocol.

If you supply a true value for the \var{implicit} flag, the returned protocol
will also adapt objects that have the specified methods or attributes.  In
other words, \code{protocolForType(file, ['read','close'], True)} returns a
protocol that will consider any object with \method{read} and \method{close}
methods to provide that protocol, as well as objects that explicitly support
\code{protocolForType(file, ['read','close'])}.

A more detailed description of this function's operations and purpose may be
found in section \ref{protocols-generated-type}.

(Note: this function may generate up to \code{2**len(\var{methods})} protocol
objects, so beware of using large method lists.)
\end{funcdesc}






\begin{funcdesc}{protocolForURI}{uri} \versionadded{0.9.1}
Return a protocol object that represents the supplied URI or UUID string.  It
is guaranteed that you will receive the same protocol object if you call this
routine more than once with equal strings.  This behavior is preserved even
across pickling and unpickling of the returned protocol object.

The purpose of this function is to permit modules to refer to protocols defined
in another module, that may or may not be present at runtime.  A more detailed
description of this function's operations and purpose may be found in
section \ref{protocols-generated-uri}.
\end{funcdesc}

\begin{funcdesc}{sequenceOf}{protocol} \versionadded{0.9.1}
Return a protocol object that represents a sequence of objects adapted to
\var{protocol}.  Thus, \code{protocols.sequenceOf(IFoo)} is a protocol that
represents a \class{protocols.IBasicSequence} of objects supporting the
\class{IFoo} protocol.  It is guaranteed that you will receive the same protocol
object if you call this routine more than once with the same protocol, even
across pickling and unpickling of the returned protocol object.

When this function creates a new sequence protocol, it automatically declares
an adapter function from \class{protocols.IBasicSequence} to the new protocol.
The adapter function returns the equivalent of \code{[adapt(x,protocol) for x
in sequence]}, unless one of the adaptations fails, in which case it returns
\code{None}, causing the adaptation to fail.

The built-in \class{list} and \class{tuple} types are declared as
implementations of \class{protocols.IBasicSequence}, so protocols returned by
\function{sequenceOf()} can be used immediately to convert lists or tuples into
lists of objects supporting \var{protocol}.  If you need to adapt other kinds
of sequences using your \function{sequenceProtocol()}, you will need to declare
that those sequences implement \class{protocols.IBasicSequence} unless they
subclass \class{tuple}, \class{list}, or some other type that implements
\class{protocols.IBasicSequence}.
\end{funcdesc}






\begin{classdesc}{StickyAdapter}{ob, proto}
\versionadded{0.9.1}
This base class is the same as the \class{Adapter} class, but with an extra
feature.  When a \class{StickyAdapter} instance is created, it declares itself
as an adapter for its \member{subject}, so that subsequent \function{adapt()}
calls will return the same adapter instance.  (Technically, it declares an
adapter function that returns itself.)

This approach is useful when an adapter wants to hold information on behalf of
its subject, that must not be lost when the subject is adapted in
more than one place.

Note that for a \class{StickyAdapter} subclass to be useful, the types it
adapts \emph{must} support \class{IOpenProvider}.  See section
\ref{protocols-instances}, ``Protocol Declarations for Individual Objects'' for
more information on this.  Also, you should never declare that a
\class{StickyAdapter} subclass adapts an individual object (as opposed to a
type or protocol), since such a declaration would create a conflict when the
adapter instance tries to register itself as an adapter for that same object
and protocol.

\class{StickyAdapter} adds one attribute to those defined by \class{Adapter}:

\begin{memberdesc}{attachForProtocols}
A tuple of additional protocols to be declared by the constructor, in
addition to the protocol the adapter class was invoked by.  Define this in
your subclass' body if it is providing multiple protocols, and wants to
use the same instance to supply them.
\end{memberdesc}

\end{classdesc}









\newpage
\subsubsection{Classes and Functions typically used for Customization/Extension}

These classes and functions are also available from the top-level
\module{protocols} package.  In contrast to the items already covered, these
classes and functions are generally needed only when extending the protocols
framework, as opposed to merely using it.

\begin{classdesc*}{Protocol}
\class{Protocol} is a base class that implements the \class{IOpenProtocol}
interface, supplying internal adapter registries for adapting from other
protocols or types/classes.  Note that you do not necessarily need to use this
class (or any other \class{IOpenProtocol} implementation) in
your programs.  Any object that implements the simpler \class{IProtocol} or
\class{IAdaptingProtocol} interfaces may be used as protocols for the
\function{adapt()} function.  Compliance with the \class{IOpenProtocol}
interface is only required to use the \module{protocols} declaration API.
(That is, functions whose names begin with \code{declare} or \code{advise}.)

To create protocols dynamically, you can create individual \class{Protocol}
instances, and then use them with the declaration API.  You can also subclass
\class{Protocol} to create your own protocol types.  If you override
\function{__init__}, however, be sure to call \function{Protocol.__init__()}
in your subclass' \function{__init__} method.
\end{classdesc*}

\begin{classdesc}{Variation}{baseProtocol \optional{, context=None}}
\versionadded{0.9.1}
A \class{Variation} is a \class{Protocol} that "inherits" adapter declarations
from an existing protocol.  When you create a \class{Variation}, it declares
that it is implied by its \var{baseProtocol}, and so any adpater suitable for
adapting to the base protocol is therefore suitable for the \class{Variation}.
This allows you to then declare adapters to the variation protocol, without
affecting those declared for the base protocol.  In this way, you can have a
protocol object that represents the use of the base protocol in a particular
context.  You can optionally specify that context via the \var{context}
argument, which will then serve as the \member{context} attribute of the
protocol.  For more background on how this works and what it might be used for,
see section \ref{protocols-context}.
\end{classdesc}


\begin{classdesc}{AbstractBaseMeta}{name, bases, dictionary}\versionadded{0.9.3}
\class{AbstractBaseMeta}, a subclass of \class{Protocol} and \class{type}, is a
metaclass used to create new "ABC-style" protocol objects, using class
statements.  You can use this metaclass directly, but it's generally simpler to
just subclass \class{AbstractBase} instead.  Normally, you will only use
\class{AbstractBaseMeta} if you need to combine it with another metaclass.
\end{classdesc}

\begin{classdesc}{InterfaceClass}{name, bases, dictionary}
\class{InterfaceClass} is a subclass of \class{AbstractBaseMeta} that implements
the convenience adapation API (see section \ref{protocols-calling}) for its
instances.  This metaclass is used to create new "pure-style" interfaces (i.e.,
protocol objects) using class statements.  Normally, you will only use
\class{InterfaceClass} directly if you need to combine it with another
metaclass, as it is usually easier just to subclass \class{Interface}.
\end{classdesc}

\begin{classdesc}{IBasicSequence} \versionadded{0.9.1}
This interface represents the ability to iterate over a container-like object,
such as a list or tuple.  An \class{IBasicSequence} object must have an
\method{__iter__()} method.  By default, only the built-in \class{list} and
\class{tuple} types are declared as having instances providing this interface.
If you want to be able to adapt to \function{sequenceOf()} protocols from other
sequence types, you should declare that their instances support this protocol.
\end{classdesc}

\begin{classdesc*}{ProviderMixin}
If you have a class with a \method{__conform__} method for its instances, but
you also want the instances to support \class{IOpenprovider} (so that
\function{adviseObject} can be used on them), you may want to include this
class as one of your class' bases.  The default adapters for
\class{IOpenprovider} can only adapt objects that do not already have a
\method{__conform__} method of their own.

So, to support \class{IOpenprovider} with a custom \method{__conform__}
method, subclass \class{ProviderMixin}, and have your \method{__conform__}
method invoke the base \method{__conform__} method as a default, using
\function{supermeta()}.  (E.g. \code{return
supermeta(MyClass,self).__conform__(protocol)}.)  See below for more on
the \function{supermeta()} function.
\end{classdesc*}
\begin{funcdesc}{metamethod}{func}
Wrap \var{func} in a manner analagous to \function{classmethod} or
\function{staticmethod}, but as a metaclass-level method that may be redefined
by metaclass instances for their instances.  For example, if a metaclass wants
to define a \method{__conform__} method for its instances (i.e. classes), and
those instances (classes) want to define a \method{__conform__} method for
\emph{their} instances, the metaclass should wrap its \method{__conform__}
method with \function{metamethod}.  Otherwise, the metaclass'
\method{__conform__} method will be hidden by the class-level
\method{__conform__} defined for the class' instances.
\end{funcdesc}

\begin{funcdesc}{supermeta}{typ, ob}
Emulates the Python built-in \function{super()} function, but with support for
metamethods.  If you ordinarily would use \function{super()}, but are calling a
\function{metamethod}, you should use \function{supermeta()} instead.  This is
because Python 2.2 does not support using super with properties (which is
effectively what metamethods are).

Note that if you are subclassing \class{ProviderMixin} or \class{Protocol}, you
will need to use \function{supermeta()} to call almost any inherited methods,
since most of the methods provided are wrapped with \function{metamethod()}.
\end{funcdesc}

\begin{funcdesc}{declareAdapterForType}{protocol, adapter, typ \optional{,
depth=1}}
Declare that \var{adapter} adapts instances of class or type \var{typ}
to \var{protocol}, by adapting \var{protocol} to \class{IOpenProtocol} and
calling its \function{registerImplementation} method.  If \var{typ} is adaptable
to \class{IOpenImplementor}, its \function{declareClassImplements} method is
called as well.
\end{funcdesc}

\begin{funcdesc}{declareAdapterForObject}{protocol, adapter, ob \optional{,
depth=1}}
Declare that \var{adapter} adapts the object \var{ob} to \var{protocol}, by
adapting \var{protocol} to \class{IOpenProtocol} and calling its
\function{registerObject} method.  Typically, \var{ob} must support
\class{IOpenProvider}.  See section \ref{protocols-instances} for details.
\end{funcdesc}

\begin{funcdesc}{declareAdapterForProtocol}{protocol, adapter, proto \optional{,
depth=1}}
Declare that \var{adapter} adapts objects that provide protocol \var{proto}
to \var{protocol}, by calling
\code{adapt(\var{proto},IOpenProtocol).addImpliedProtocol(\var{protocol},\var{adapter},\var{depth})}.

\end{funcdesc}

































\newpage
\subsubsection{\module{protocols.interfaces} --- Package Interfaces\label{protocols-interfaces-module}}
\declaremodule{}{protocols.interfaces}

\note{All of the interfaces listed here can also be imported directly from the
top-level \module{protocols} package.  However, you will probably only need
them if you are extending the framework, as opposed to merely using it.}

\begin{classdesc*}{IOpenProtocol}
This interface documents the behavior required of protocol objects in order to
be used with the \module{protocols} declaration API (the functions whose names
begin with \code{declare} or \code{advise}.)  The declaration API functions
will attempt to \function{adapt()} supplied protocols to this interface.

The methods an \class{IOpenProtocol} implementation must supply are:

\begin{methoddesc}{addImpliedProtocol}{proto, adapter, depth}
Declare that this protocol can be adapted to protocol \var{proto} via the
\class{IAdapterFactory} supplied in \var{adapter}, at the specified implication
level \var{depth}.  The protocol object should ensure that the implied protocol
is able to adapt objects implementing its protocol (typically by recursively
invoking \function{declareAdapterForType()} with increased depth and
appropriately composed adapters), and notify any registered implication
listeners via their \method{newProtocolImplied()} methods.  If the protocol
already implied \var{proto}, this method should have no effect and send no
notifications unless the new \var{adapter} and \var{depth} represent a
``shorter path'' as described in section \ref{proto-implication}.
\end{methoddesc}

\begin{methoddesc}{registerImplementation}{klass, adapter, depth}
Declare that instances of type or class \var{klass} can be adapted to this
protocol via the \class{IAdapterFactory} supplied in \var{adapter}, at the
specified implication level \var{depth}.  Unless \var{adapter} is
\function{DOES_NOT_SUPPORT}, the protocol object must ensure that
any protocols it implies are also able to perform the adaptation (typically
by recursively invoking \function{declareAdapterForType()} with increased depth
and appropriately composed adapters for its implied protocols).  If the
protocol already knew a way to adapt instances of \var{klass}, this method
should be a no-op unless the new \var{adapter} and \var{depth} represent a
``shorter path'' as described in section \ref{proto-implication}.
\end{methoddesc}

\begin{methoddesc}{registerObject}{ob, adapter, depth}
Ensure that the specific object \var{ob} will be adapted to this protocol via
the \class{IAdapterFactory} supplied in \var{adapter}, at the specified
implication level \var{depth}.  The protocol object must also ensure that the
object can be adapted to any protocols it implies.  This method may be
implemented by adapting \var{ob} to \class{IOpenProvider}, calling the
\method{declareProvides()} method, and then recursively invoking
\function{declareAdapterForObject} with increased depth and appropriately
composed adapters for the protocols' implied protocols.
\end{methoddesc}

\begin{methoddesc}{addImplicationListener}{listener}
Ensure that \var{listener} (an \class{IImplicationListener}) will be notified
whenever an implied protocol is added to this protocol, or an implication path
from this protocol is shortened.  The protocol should at most retain a weak
reference to \var{listener}.  Note that if a protocol can guarantee that no
notices will ever need to be sent, it is free to implement this method as a
no-op.  For example, Zope interfaces cannot imply any protocols besides their
base interfaces, which are not allowed to change.  Therefore, no change
notifications would ever need to be sent, so the \class{IOpenProtocol} adapter
for Zope interfaces implements this method as a no-op.
\end{methoddesc}


\note{\class{IOpenProtocol} is a subclass of \class{IAdaptingProtocol}, which
means that implementations must therefore meet its requirements as well, such
as having an \method{__adapt__()} method.}

\end{classdesc*}












\begin{classdesc*}{IOpenProvider}
This interface documents the behavior required of an object to be usable with
\function{adviseObject()}.  Note that some protocol objects, such as the
\class{IOpenProtocol} adapter for Zope interfaces, can handle
\function{adviseObject()} operations without adapting the target object to
\class{IOpenProvider}.  This should be considered an exception, rather than the
rule.  However, the \module{protocols} package declares default adapters so
that virtually any Python object that doesn't already have a
\method{__conform__()} method can be adapted to \class{IOpenProvider}
automatically.

\begin{methoddesc}{declareProvides}{protocol, adapter, depth}
Declare that this object can provide \var{protocol} if adapted by the
\class{IAdapterFactory} supplied in \var{adapter}, at implication level
\var{depth}.  Return a true value if the new adapter was used, or a false
value if the object already knew a ``shorter path'' for adapting to
\var{protocol} (as described in section \ref{proto-implication}).  Typically,
an implementation of this method will also adapt \var{protocol} to
\class{IOpenProtocol}, and then register with \method{addImplicationListener()}
to receive notice of any protocols that might be implied by \var{protocol}
in future.
\end{methoddesc}

\end{classdesc*}

















\begin{classdesc*}{IImplicationListener}
This interface documents the behavior required of an object supplied to
the \method{IOpenProtocol.addImplicationListener()} method.  Such objects must
be weak-referenceable, usable as a dictionary key, and supply the following
method:

\begin{methoddesc}{newProtocolImplied}{srcProto, destProto, adapter, depth}
Receive notice that an adaptation was declared from \var{srcProto} to
\var{destProto}, using the \class{IAdapterFactory} \var{adapter}, at
implication level \var{depth}.

When used as part of an \class{IOpenProvider} implementation, this method is
typically used to recursively invoke \function{declareAdapterForObject()} with
increased depth and appropriately composed adapters from protocols already
supported by the object.
\end{methoddesc}

\end{classdesc*}























\begin{classdesc*}{IOpenImplementor}
If an class or type supplied to \function{declareAdapterForType} supports
this interface, it will be notified of the declaration and any future
declarations that affect the class, due to current or future protocol
implication relationships.  Supporting this interface is not necessary; it is
provided as a hook for advanced users.  Note that to declare a class or type
as an \class{IOpenImplementor}, you must call \code{adviseObject(theClass,
provides=[IOpenImplementor])} after the class definition or place
\code{advise(classProvides=[IOpenImplementor])} in the body of the class, since
this interface must be provided by the class itself, not by its instances.  (Of
course, if you implement this interface via a metaclass, you can declare that
the metaclass' instances provide the interface.)

Notification to classes supporting \class{IOpenImplementor} occurs via the
following method:

\begin{methoddesc}{declareClassImplements}{protocol, adapter, depth}
Receive notice that instances of the class support \var{protocol} via the
the \class{IAdapterFactory} supplied in \var{adapter}, at implication level
\var{depth}.
\end{methoddesc}

\end{classdesc*}


















\begin{classdesc*}{IAdapterFactory}
An interface documenting the requirements for an object to be used as an
adapter factory: i.e., that it be a callable accepting an object and a
protocol.)  This interface is not used by the \module{protocols} package except
as documentation.
\end{classdesc*}


\begin{classdesc*}{IProtocol}
An interface documenting the basic requirements for an object to be used as
a protocol for \function{adapt()}: i.e., that it be usable as a dictionary
key.  This interface is not used by the \module{protocols} package except as
documentation.
\end{classdesc*}


\begin{classdesc*}{IAdaptingProtocol}
An interface documenting the requirements for a protocol object to be able to
adapt objects when used with \function{adapt()}: i.e., that it have a
\method{__adapt__} method that accepts the object to be adapted and returns
either an object providing the protocol or \constant{None}.  This interface is
not used by the \module{protocols} package except as documentation.  It is a
subclass of \class{IProtocol}, so any implementation of this interface must
support the requirements defined by \class{IProtocol} as well.
\end{classdesc*}















\newpage
\subsubsection{\module{protocols.adapters} --- ``Adapter arithmetic'' support \label{protocol-adapters-module}}
\declaremodule{}{protocols.adapters}

The \module{protocols.adapters} module provides support for doing ``adapter
arithmetic'' such as determining which of two adapter paths is shorter,
composing a new adapter from two existing adapters, and updating an adapter
registry with a new adapter path.  See section \ref{proto-implication} for a
more general discussion of adapter arithmetic.

\begin{funcdesc}{minimumAdapter}{a1, a2 \optional{, d1=0, d2=0}}
Find the ``shortest'' adapter path, \var{a1} at depth \var{d1}, or \var{a2} at
depth \var{d2}.  Assuming \var{a1} and \var{a2} are adapter factories that
accept similar input and return similar output, this function returns the one
which is the ``shortest path'' between its input and its output.  That is, the
one with the smallest implication depth (\var{d1} or \var{d2}), or, if the
depths are equal, then the adapter factory that is composed of the fewest
chained factories (as composed by \function{composeAdapters()}) is returned.
If neither factory is composed of multiple factories, or they are composed of
the same number of intermediate adapter factories, then the following
preference order is used:

\begin{enumerate}
\item If one of the adapters is \function{NO_ADAPTER_NEEDED}, it is returned
\item If one of the adapters is \function{DOES_NOT_SUPPORT}, the \emph{other}
adapter is returned.
\item If both adapters are the exact same object (i.e. \code{a1 is a2}), either
one is returned
\end{enumerate}

If none of the above conditions apply, then the adapter precedence is
considered ambiguous, and a \exception{TypeError} is raised.

This function is used by \function{updateWithSimplestAdapter} to determine
whether a new adapter declaration should result in a registry update.  Note
that the determination of adapter composition length uses the
\member{__adapterCount__} attribute, if present.  (It is assumed to be
\constant{1} if not present.  See \function{composeAdapters()} for more
details.)
\end{funcdesc}


\begin{funcdesc}{composeAdapters}{baseAdapter, baseProtocol, extendingAdapter}
Return a new \class{IAdapterFactory} composed of the input adapter factories
\var{baseAdapter} and \var{extendingAdapter}.  If either input adapter is
\function{DOES_NOT_SUPPORT}, \function{DOES_NOT_SUPPORT} is returned.  If
either input adapter is \function{NO_ADAPTER_NEEDED}, the other input adapter
is returned.  Otherwise, a new adapter factory is created that will return
\code{\var{extendingAdapter}(\var{baseAdapter}(object,\var{baseProtocol}),protocol)}
when called with an \code{object, protocol} pair.  (Note: the actual
implementation verifies that \var{baseAdapter} didn't return \constant{None}
before it calls \var{extendingAdapter}).

If this function creates a new adapter factory, the factory will have an
\member{__adapterCount__} attribute set to the sum of the
\member{__adapterCount__} attributes of the input adapter factories.  If an
input factory does not have an \member{__adapterCount__} attribute, it is
assumed to equal \constant{1}.  This is done so that the
\function{minimumAdapter()} can compare the length of composed adapter chains.
\end{funcdesc}


\begin{funcdesc}{updateWithSimplestAdapter}{mapping, key, adapter, depth}
Treat \var{mapping} as an adapter registry, replacing the entry designated by
\var{key} with an \code{(\var{adapter},\var{depth})} tuple, if and only if
the new entry would be a ``shorter path'' than the existing entry, if any.
(I.e., if \code{minimumAdapter(old, \var{adapter}, oldDepth, \var{depth})}
returns \var{adapter}, and \var{adapter} is not the existing registered
adapter.  The function returns a true value if it updates the contents of
\var{mapping}.

This function is used to manage type-to-protocol, protocol-to-protocol, and
object-to-protocol adapter registries, keyed by type or protocol.  The
\var{mapping} argument must be a mapping providing \method{__setitem__()}
and \method{get()} methods.  Values stored in the mapping will be
\code{(\var{adapter},\var{depth})} tuples.
\end{funcdesc}





\newpage
\subsubsection{\module{protocols.zope_support} --- Support for Zope Interfaces}
\declaremodule[protocols.zopesupport]{}{protocols.zope_support}

Importing this module enables experimental support for using Zope X3
\class{Interface} objects with the \module{protocols} package, by registering
an adapter from Zope X3's \class{InterfaceClass} to \class{IOpenProtocol}.  The
adapter supports the following subset of the declaration API:

\begin{itemize}

\item The only adapters supported via Zope APIs are \function{NO_ADAPTER_NEEDED}
and \function{DOES_NOT_SUPPORT}.  By using PyProtocols APIs, you may declare and
use other adapters for Zope interfaces, but Zope itself will not use them, since
the Zope interface API does not directly support adaptation.

\item Zope's interface APIs do not conform to \module{protocols} package
``shortest path wins'' semantics.  Instead, new declarations override older
ones.

\item Interface-to-interface adaptation may not work if a class only declares
what it implements using Zope's interface API.  That is, if a class declares
that it implements \class{ISomeZopeInterface}, and you define an adaptation from
\class{ISomeZopeInterface} to \class{ISomeOtherInterface}, PyProtocols may not
recognize that the class can be adapted to \class{ISomeOtherInterface}.

\item Changing the \member{__bases__} of a class that has Zope interfaces
declared for it (either as ``class provides'' or ``instances provide''), may
have unexpected results, because Zope uses inheritance of a single descriptor
to control declarations.  In general, it will only work if the class whose
\member{bases} are changed, has no declarations of its own.

\item You cannot declare an implication relationship from a Zope
\class{Interface}, because Zope only supports implication via
inheritance, which is fixed at interface definition time.  Therefore, you cannot
create a ``subset'' of a Zope \class{Interface}, and subscribing an
\class{IImplicationListener} to an adapted Zope \class{Interface}
silently does nothing.

\item You can, however, declare that a \module{protocols.Interface} extends a
Zope \class{Interface}.  Declaring that a class' instances or that an object
provides the extended interface, will automatically declare that the class'
instances or the object provides the Zope \class{Interface} as well.  For
example:

\begin{verbatim%
}import protocols
from zope.somepackage.interfaces import IBase

class IExtended(protocols.Interface):
    advise(
        protocolExtends = [IBase]
    )

class AnImplementation:
    advise(
        instancesProvide = [IExtended]
    )
\end{verbatim}

The above code should result in Zope recognizing that instances of
\class{AnImplementation} provide the Zope \class{IBase} interface.

\item You cannot extend both a Zope interface and a Twisted interface in the
same \class{protocols.Interface}.  Although this may not give you any errors,
Twisted and Zope both expect to use an \member{__implements__} attribute to
store information about what interface a class or object provides.  But each has
a different interpretation of the contents, and does not expect to find
``foreign'' interfaces contained within.  So, until this issue between Zope and
Twisted is resolved, it is not very useful to create interfaces that extend
both Zope and Twisted interfaces.

\item Zope does not currently appear to support classes inheriting direct
declarations (e.g. \code{classProvides}).  This appears to be a by-design
limitation.

\end{itemize}

The current implementation of support for Zope X3 interfaces is currently based
on Zope X3 alpha 1; it will not work with older releases.  Zope
X3 requires Python 2.3.3 or better, so even though PyProtocols works
with 2.2.2 and up in general, you will need 2.3.3 to use PyProtocols with Zope X3.
\newpage
\subsubsection{\module{protocols.twisted_support} --- Support for Twisted Interfaces}
\declaremodule[protocols.twistedsupport]{}{protocols.twisted_support}

Importing this module enables experimental support for using Twisted 1.1.0
\class{Interface} objects with the \module{protocols} package, by registering
an adapter from Twisted's \class{MetaInterface} to \class{IOpenProtocol}.  The
adapter supports the following subset of the declaration API:

\begin{itemize}

\item Only protocol-to-protocol adapters defined via the \module{protocols}
declaration API will be available to implication listeners.  If
protocol-to-protocol adapters are registered via Twisted's
\function{registerAdapter()}, implication listeners are \emph{not} notified.

\item You cannot usefully create a ``subset'' of a Twisted interface, or an
adaptation from a Twisted interface to another interface type, as Twisted
insists that interfaces must subclass its interface base class.  Also, Twisted
does not support transitive adaptation, nor can it notify the destination
interface(s) of any new incoming adapter paths.

\item If you register an adapter factory that can return \constant{None} with
a Twisted interface, note that Twisted does not check for a \class{None} return
value from \function{getAdapter()}.  This means that code in Twisted might
receive \constant{None} when it expected either an implementation or an error.

\item Only Twisted's global adapter registry is supported for declarations and
\function{adapt()}.

\item Twisted doesn't support classes providing interfaces (as opposed to their
instances providing them).  You may therefore obtain unexpected results if you
declare that a class provides a Twisted interface or an interface that extends a
Twisted interface.

\item Changing the \member{__bases__} of a class that has Twisted interfaces
declared for it may have unexpected results, because Twisted uses inheritance of
a single descriptor to control declarations.  In general, it will only work if
the class whose \member{bases} are changed, has no declarations of its own.



\item Any adapter factory may be used for protocol-to-protocol adapter
declarations.  But, for any other kind of declaration,
\function{NO_ADAPTER_NEEDED} and \function{DOES_NOT_SUPPORT} are the only
adapter factories that can be used with Twisted.

\item Twisted interfaces do not conform to \module{protocols} package ``shortest
path wins'' semantics.  For protocol-to-protocol adapter declarations, only one
adapter declaration between a given pair of interfaces is allowed.  Any
subsequent declarations with the same source and destination will result in
a \exception{ValueError}.  For all other kinds of adapter declarations, new
declarations override older ones.

\item You cannot extend both a Zope interface and a Twisted interface in the
same \class{protocols.Interface}.  Although this may not give you any errors,
Twisted and Zope both expect to use an \member{__implements__} attribute to
store information about what interface a class or object provides.  But each has
a different interpretation of the contents, and does not expect to find
``foreign'' interfaces contained within.  So, until this issue between Zope and
Twisted is resolved, it is not very useful to create interfaces that extend
both Zope and Twisted interfaces.

\end{itemize}


















\newpage
\subsubsection{\module{protocols.advice} --- Metaclasses and other ``Magic''}
\declaremodule{}{protocols.advice}

This module provides a variety of utility functions and classes used by the
\module{protocols} package.  None of them are really specific to the
\module{protocols} package, and so may be useful to other libraries or
applications.

\begin{funcdesc}{addClassAdvisor}{callback \optional{, depth=\constant{2}}}
Set up \var{callback} to be called with the containing class, once it is
created.  This function is designed to be called by an ``advising'' function
(such as \function{protocols.advise()}) executed in the body of a class suite.
The ``advising'' function supplies a callback that it wishes to have executed
when the containing class is created.  The callback will be given one argument:
the newly created containing class.  The return value of the callback will be
used in \emph{place} of the class, so the callback should return the input if
it does not wish to replace the class.

The optional \var{depth} argument determines the number of frames between this
function and the targeted class suite.  \var{depth} defaults to 2, since this
skips this function's frame and one calling function frame.  If you use this
function from a function called directly in the class suite, the default will
be correct, otherwise you will need to determine the correct depth yourself.

This function works by installing a special class factory function in
place of the \member{__metaclass__} of the containing class.  Therefore, only
callbacks \emph{after} the last \member{__metaclass__} assignment in the
containing class will be executed.  Be sure that classes using ``advising''
functions declare any \member{__metaclass__} \emph{first}, to ensure all
callbacks are run.
\end{funcdesc}

\begin{funcdesc}{isClassAdvisor}{ob}
Returns truth if \var{ob} is a class advisor function.  This is used to
determine if a \member{__metaclass__} value is a ``magic'' metaclass installed
by \function{addClassAdvisor()}.  If so, then \var{ob} will have a
\member{previousMetaclass} attribute pointing to the previous metaclass,
if any, and a \member{callback} attribute containing the callback that was
given to \function{addClassAdvisor()}.
\end{funcdesc}

\begin{funcdesc}{getFrameInfo}{frame}
Return a \code{(\var{kind},\var{module},\var{locals},\var{globals})} tuple for
the supplied frame object.  The returned \var{kind} is a string: either
``exec'', ``module'', ``class'', ``function call'', or ``unknown''.
\var{module} is the module object the frame is/was executed in, or
\constant{None} if the frame's globals could not be correlated with a module in
\code{sys.modules}.  \var{locals} and \var{globals} are the frame's local
and global dictionaries, respectively.  Note that they can be the same
dictionary, and that modifications to locals may not have any effect on the
execution of the frame.

This function is used by functions like \function{addClassAdvisor()} and
\function{advise()} to verify where they're being called from, and to work
their respective magics.
\end{funcdesc}


\begin{funcdesc}{getMRO}{ob \optional{, extendedClassic=\constant{False}}}
Return an iterable over the ``method resolution order'' of \var{ob}.  If
\var{ob} is a ``new-style'' class or type, this returns its \member{__mro__}
attribute.  If \var{ob} is a ``classic'' class, this returns
\code{classicMRO(\var{ob},\var{extendedClassic})}.  If \var{ob} is not a class
or type of any kind, a one-element sequence containing just \var{ob} is
returned.
\end{funcdesc}


\begin{funcdesc}{classicMRO}{ob \optional{, extendedClassic=\constant{False}}}
Return an iterator over the ``method resolution order'' of classic class
\var{ob}, following the ``classic'' method resolution algorithm of recursively
traversing \member{__bases__} from left to right.  (Note that this may return
the same class more than once, for some inheritance graphs.)  If
var{extendedClassic} is a true value, \class{InstanceType} and \class{object}
are added at the end of the iteration.  This is used by \class{Protocol}
objects to allow generic adapters for \class{InstanceType} and \class{object}
to be used with ``classic'' class instances.
\end{funcdesc}




\begin{funcdesc}{determineMetaclass}{bases \optional{, explicit_mc=\constant{None}}}
Determine the metaclass that would be used by Python, given a non-empty
sequence of base classes, and an optional explicitly supplied
\member{__metaclass__}.  Returns \class{ClassType} if all bases are ``classic''
and there is no \var{explicit_mc}.  Raises \exception{TypeError} if the bases'
metaclasses are incompatible, just like Python would.
\end{funcdesc}


\begin{funcdesc}{minimalBases}{classes}
Return the shortest ordered subset of the input sequence \var{classes} that
still contains the ``most specific'' classes.  That is, the result sequence
contains only classes that are not subclasses of each other.  This function is
used by \function{determineMetaclass()} to narrow down its list of candidate
metaclasses, but is also useful for dynamically generating metaclasses.
\end{funcdesc}



\begin{funcdesc}{mkRef}{ob \optional{, callable}}
If \var{ob} is weak-referenceable, returns
\code{weakref.ref(\var{ob},\var{callable})}.  Otherwise, returns a
\code{StrongRef(\var{ob})}, emulating the interface of \code{weakref.ref()}.
This is used by code that wants to use weak references, but may be given
objects that are not weak-referenceable.  Note that \var{callable}, if
supplied, will \emph{not} be called if \var{ob} is not weak-referenceable.
\end{funcdesc}

\begin{classdesc}{StrongRef}{ob}
An object that emulates the interface of \class{weakref.ref()}.  When called,
an instance of \class{StrongRef} will return the \var{ob} it was created for.
Also, it will hash the same as \var{ob} and compare equal to it.  Thus, it
can be used as a dictionary key, as long as the underlying object can.  Of
course, since it is not really a weak reference, it does not contribute to the
garbage collection of the underlying object, and may in fact hinder it, since
it holds a live reference to the object.
\end{classdesc}



\newpage
\subsection{Big Example 2 --- Extending the Framework for Context\label{protocols-context}}

Now it's time for our second ``big'' example.  This time, we're going to add
an extension to the \module{protocols} framework to support ``contextual
adaptation''.  The tools we've covered so far are probably adequate to support
80-90\% of situations requiring adaptation.  But, they are essentially global
in nature: only one adapter path is allowed between any two points.  What if we
need to define a different adaptation in a specific context?

For example, let's take the documentation framework we began designing in
section \ref{protocols-example1}.  Suppose we'd like, for the duration of a
single documentation run, to replace the factory that adapts from
\class{FunctionType} to \class{IDocumentable}?  For example, we might like to
do this so that functions used by our ``finite state machine'' objects as
``transitions'' are documented differently than regular functions.

Using only the tools described so far, we can't do this if
\class{IDocumentable} is a single object.  The framework that registered the
\class{FunctionAsDocumentable} adapter effectively ensured that we cannot
replace that adapter with another, since it is already the shortest adapter
path.  What can we do?

In section \ref{proto-implication}, we discussed how we could create ``subset''
protocols and ``inherit'' adapter declarations from existing protocols.  In
this way, we could create a new subset protocol of \class{IDocumentable}, and
then register our context-specific adapters with that subset.  These subset
protocols are just as fast as the original protocols in looking up adapters, so
there's no performance penalty.

But who creates the subset protocol?  The client or the framework?  And how do
we get the framework to use our subset instead of its built-in
\class{IDocumentable} protocol?

To answer these questions, we will create an extension to the
\module{protocols} framework that makes it easy for frameworks to manage
``contextual'' or ``local'' protocols.  Then, framework creators will have a
straightforward way to support context-specific adapter overrides.




As before, we'll start by envisioning our ideal situation.  Let's assume that
our documentation tools are object-based.  That is, we instantiate a
``documentation set'' or ``documentation run'' object in order to generate
documentation.  How do we want to register adapters?  Well, we could have the
framework add a bunch of methods to do this, but it seems more straightforward
to simply supply the interfaces as attributes of the ``documentation set'' or
``documentation run'' object, e.g.:

\begin{verbatim%
}from theDocTool import DocSet
from myAdapters import specialFunctionAdapter
from types import FunctionType
import protocols

myDocs = DocSet()

protocols.registerAdapter(
    specialFunctionAdapter,
    provides = [myDocs.IDocumentable],
    forTypes = [FunctionType]
)

myDocs.run()
\end{verbatim}

So, instead of importing the interface, we access it as an attribute of some
relevant ``context'' object, and declare adapters for it.  Anything we don't
declare a ``local'' adapter for, will use the adapters declared for the
underlying ``global'' protocol.

Naturally, the framework author could implement this by writing code in the
\class{DocSet} class' \method{__init__} method, to create the new
``local'' protocol and register it as a subset of the ``global''
\class{IDocumentable} interface.  But that would be time-consuming and error
prone, and therefore discourage the use of such ``local'' protocols.

Again, let's consider what our ideal situation would be.  The author of the
\class{DocSet} class should be able to do something like:



\begin{verbatim%
}class DocSet:

    from doctool.interfaces import IDocumentable, ISignature

    IDocumentable = subsetPerInstance(IDocumentable)
    ISignature = subsetPerInstance(ISignature)

    # ... etc.
\end{verbatim}

Our hypothetical \class{subsetPerInstance} class would be a descriptor that
did all the work needed to provide a ``localized'' version of each interface
for each instance of \class{DocSet}.  Code in the \class{DocSet} class would
always refer to \code{self.IDocumentable} or \code{self.ISignature}, rather
than using the ``global'' versions of the interfaces.  Thus, we can now
register adapters that are unique to a specific \class{DocSet}, but still use
any globally declared adapters as defaults.

Okay, so that's our hypothetical ideal.  How do we implement it?  I personally
like to try writing the ideal thing, to find out what other pieces are needed.
So let's start with writing the \class{subsetPerInstance} descriptor, since
that's really the only piece we know we need so far.


















\begin{verbatim%
}from protocols import Protocol, declareAdapterForProtocol, NO_ADAPTER_NEEDED

class subsetPerInstance(object):

    def __init__(self,protocol,name=None):

        self.protocol = protocol
        self.name = name or getattr(protocol,'__name__',None)

        if not self.name:
            raise TypeError("Descriptor needs a name for", protocol)


    def __get__(self,ob,typ=None):

        if ob is None:
            return self

        name = self.name
        if getattr(type(ob),name) is not self or name in ob.__dict__:
            raise TypeError(
                "Descriptor is under more than one name or the wrong name",
                self, name, type(ob)
            )

        local = Protocol()
        declareAdapterForProtocol(local,NO_ADAPTER_NEEDED,self.protocol)

        # save it in the instance's dictionary so we won't be called again
        ob.__dict__[name] = local
        return local


    def __repr__(self):
        return "subsetPerInstance(%r)" % self.protocol
\end{verbatim}




Whew.  Most of the complexity above comes from the need for the descriptor to
know its ``name'' in the containing class.  As written, it will guess its name
to be the name of the wrapped interface, if available.  It can also detect
some potential aliasing/renaming issues that could occur.  The actual work of
the descriptor occurs in just two lines, buried deep in the middle of the
\method{__get__} method.

As written, it's a handy enough tool.  We could leave things where they are
right now and still get the job done.  But that would hardly be an example of
extending the framework, since we didn't even subclass anything!

So let's add another feature.  As it sits, our descriptor should work with both
old and new-style classes, automatically generating one subset protocol for
each instance of its containing class.  But, the subset protocol doesn't
\emph{know} it's a subset protocol, or of what context.  If we were to print
\code{DocSet().IDocumentable}, we'd just get something like
\constant{<protocols.interfaces.Protocol instance at 0x00ABA220>}.

Here's what we'd like it to do instead.  We'd like it to say something like
\constant{LocalProtocol(<class 'IDocumentable'>, <DocSet instance at
0x00AD9FB0>)}.  That is, we want the local protocol to:

\begin{itemize}
\item ``know'' it's a local protocol
\item know what protocol it's a local subset of
\item know what ``context'' object it's a local protocol for
\end{itemize}

What does this do for us?  Aside from debugging, it gives us a chance to find
related interfaces, or access methods or data available from the context.
Let's suppose our \class{specialFunctionAdapter} class needs to be able to
adapt to \class{ISignature}.  If it uses the global \class{ISignature}, it
won't have access to any context-specific adapters for \class{ISignature}.  How
can we get it to use the \emph{local} \class{ISignature} instead?

Remember that every adapter factory is passed two arguments: the component to
adapt, and the protocol to adapt it to.  If the protocol is a local one, and it
knows its ``context'' object, then the adapter could access other interfaces by
retrieving them from the context.  (Along with anything else it needs from
the context!)

So, let's create a \class{LocalProtocol} class:

\begin{verbatim%
}class LocalProtocol(Protocol):

    def __init__(self, baseProtocol, context):

        self.baseProtocol = baseProtocol
        self.context = context

        # Note: Protocol is a ``classic'' class, so we don't use super()
        Protocol.__init__(self)

        declareAdapterForProtocol(self,NO_ADAPTER_NEEDED,baseProtocol)

    def __repr__(self):
        return "LocalProtocol(%r,%r)" % (self.baseProtocol, self.context)
\end{verbatim}

And now, we can replace these two lines in our earlier \method{__get__} method:

\begin{verbatim%
}        local = Protocol()
        declareAdapterForProtocol(local,NO_ADAPTER_NEEDED,self.protocol)
\end{verbatim}

with this one:

\begin{verbatim%
}        local = LocalProtocol(self.protocol, ob)
\end{verbatim}

Thus, the new local protocol will know its context is the instance it was
retrieved from.  Context-sensitive adapters will be able to retrieve their
context from it, and if we are debugging, we'll be able to know what the
original protocol was.





Of course, to make this new extension really robust, we would need to add some
more documentation.  For example, it might be good to add an
\class{ILocalProtocol} interface that documents what local protocols do.
Context-sensitive adapters would then be able to verify whether they are
working with a local protocol or a global one.  Framework developers would also
want to document what local interfaces are provided by their frameworks'
objects, and authors of context-sensitive adapters need to document what
interface they expect their local protocols' \member{context} attribute to
supply!  Also, see below for a web site with some interesting papers on
patterns for using localized adaptation of this kind.

\note{In practice, the idea of having local protocols turned out to be useful enough
that as of version 0.9.1, our \class{LocalProtocol} example class was added to
the protocols package as \class{protocols.Variation}.  So, if you want to
make use of the idea, you don't need to type in the source or write your own
any more.}

\begin{seealso}

\seetitle[http://www.objectteams.org/]{Object Teams}{If you find the idea of
context-specific interfaces and adapters interesting, you'll find ``Object
Teams'' intriguing as well.  In effect, the ideas we've presented here map onto
a subset of the ``Object Teams'' concept.  Our local interfaces correspond
to their ``abstract roles'', our local adapters' instances map to their ``role
instances'', and our contexts are their ``team instances''.  Adapting an object
corresponds to their ``lifting'', and so on.  The main concept that's not
directly supported by our implementation here is ``callin binding''.  (Callin
binding is a way of (possibly temporarily) injecting hooks into an adapted
object so that the adapter can be informed when the adapted object's methods
are called directly by other code.)}

\end{seealso}








\newpage
\subsection{Additional Examples and Usage Notes}

If you have any ideas or examples you'd like to share for inclusion in this
section, please contact the author.  In the meantime, here are a few additional
examples of things you can do with \function{adapt()} and the \module{protocols}
package.


\subsubsection{Double Dispatch and the ``Visitor'' Pattern\label{dispatch-example}}

Double dispatch and the ``Visitor'' pattern are mechanisms for selecting a
method to be executed, based on the type of two objects at the same time.
To implement either pattern, both object types must have code specifically to
support the pattern.  Object adaptation makes this easier by requiring at most
one of the objects to directly support the pattern; the other side can provide
support via adaptation.  This is useful both for writing new code clearly and
for adapting existing code to use the pattern.

First, let's look at double dispatching.  Suppose we are creating a business
application GUI that supports drag-and-drop.  We have various kinds of objects
that can be dragged and dropped onto other objects: users, files, folders, a
trash can, and a printer.  When we drop a user on a file, we want to grant the
user access to the file, and when we drop a file on the user, we want to email
them the file.  If we drop a file on a folder, it should be filed in the folder,
but if we drop the folder on the file, that's an error.  The classic ``double
dispatch'' approach would look something like:















\begin{verbatim%
}class Printer:
    def drop(self,thing):
        thing.droppedOnPrinter(self)

class Trashcan:
    def drop(self,thing):
        thing.droppedInTrash(self)

class User:
    def drop(self,thing):
        thing.droppedOnUser(self)

    def droppedOnPrinter(self,printer):
        printer.printUser(self)

    def droppedInTrash(self,trash):
        self.delete()

class File:
    def drop(self,thing):
        thing.droppedOnFile(self)

    def droppedOnPrinter(self,printer):
        printer.printFile(self)

    def droppedOnUser(self,user):
        user.sendMail(self)

    def droppedInTrash(self,trash):
        self.delete()
\end{verbatim}

We've left out any of the methods that actually \emph{do} anything, of course,
and all of the methods for things that the objects don't do.  For example, the
\class{Trashcan} should have methods for \method{droppedInTrash()},
\method{droppedOnPrinter()}, etc., that display an error or beep or whatever.
(Of course, in Python you can just trap the \exception{AttributeError} from the
missing method to do this; but we didn't show that here either.)


Every time another kind of object is added to this system, new \code{droppedOnX}
methods spring up everywhere like weeds.  Now let's look at the adaptation
approach:

\begin{verbatim%
}class Printer:
    def drop(self,thing):
        IPrintable(thing).printOn(self)

class Trashcan:
    def drop(self,thing):
        IDeletable(thing).delete(self)

class User:
    protocols.advise(instancesProvide=[IDeletable,IPrintable])
    def drop(self,thing):
        IMailable(thing).mailTo(self)

class File:
    protocols.advise(instancesProvide=[IDeletable,IMailable,IPrintable])
    def drop(self,thing):
        IInsertable(thing).insertInto(self)

class Undroppable(protocols.Adapter):
    protocols.advise(
        instancesProvide=[IPrintable,IDeletable,IMailable,IInsertable],
        asAdapterForTypes=[object]
    )

    def printOn(self,printer):
        print "Can't print", self.subject

    def mailTo(self,user):
        print "Can't mail", self.subject

    # ... etc.
\end{verbatim}




Notice how our default \class{Undroppable} adapter class implements the
\class{IPrintable}, \class{IDeletable}, \class{IMailable}, and
\class{IInsertable} protocols on behalf of arbitrary objects, by giving user
feedback that the operation isn't possible.  (This technique of using a default
adapter factory that provides an empty or error-raising implementation of an
interface, is an example of the \strong{null object pattern}.)

Notice that the adaptation approach is much more scalable, because new methods
are not required for every new droppable item.  Third parties can declare
adaptations between two other developers' objects, making drag and drop between
them possible.

Now let's look at the ``Visitor'' pattern.
The ``Visitor'' pattern is a specialized form of double dispatch, used to apply
an algorithm to a structured collection of objects.  For example, the Python
\program{docutils} tookit implements the visitor pattern to create various
kinds of output from a document node tree (much like an XML DOM).  Each node has
a \method{walk()} method that accepts a ``visitor'' argument.  The visitor must
provide a set of \code{visit_X} methods, where X is the name of a type of node.
The idea of the approach is that one can write new visitor types that perform
different functions.  One visitor writes out HTML, another writes out LaTeX or
maybe plain ASCII text.  The nodes don't care what the visitor does, they just
tell it what kind of object is being visited.

Like double dispatch, this pattern is definitely an improvement over writing
large if-then-else blocks to introspect types.  But it does have a few
drawbacks.  First, all the types must have unique names.  Second, the visitor
must have methods for all possible node types (or the caller must handle the
absence of the methods).  Third, there is no way for the methods to mimic the
inheritance or interface structure of the source types.  So, if there are node
types like \class{Shape} and \class{Square}, you must write \method{visit_Shape}
and \method{visit_Square} methods, even if you would like to treat all subtypes
of \class{Shape} the same.








The object adaptation approach to this, is to define visitor(s) as adapters from
the objects being traversed, to an interface that supplies the desired behavior.
For example, one might define \class{IHTMLWriter} and \class{ILaTeXWriter}
interfaces, with \class{writeHTML()} and \class{writeLaTeX()} methods.  Then, by
defining adapters from the appropriate node base types to these interfaces, the
desired behavior is achieved.  Just use
\code{IHTMLWriter(document).writeHTML()}, and off you go.

This approach is far less fragile, since new node types do not require new
methods in the visitors, and if the new node type specializes an existing type,
the default adaptation might be reasonable.  Also, the approach is non-invasive,
so it can be applied to existing frameworks that don't support the visitor
pattern (such as \module{xml.dom.minidom}).  Further, the adapters can exercise
fine-grained control over any traversal that takes place, since it is the
adapter rather than the adaptee that controls the visiting order.

Last, but not least, notice that by adapting from interfaces rather than
types, one can apply this pattern to multiple implementations of the interface.
For example, Python has many XML DOM implementations; to the extent that two
implementations provide the same interface, the adapters you write could be used
with any of them, even if each pacakge has different names for their node types.

Are there any downsides to using adaptation over double-dispatch or the Visitor
pattern?  The total size of your program may be larger, because you'll be
writing lots of adapter classes.  But, your program will also be more modular,
and you'll be able to group the classes in ways that make more sense for the
reader.  Using adaptation also may be faster or slower than not using it,
depending on various implementation factors.

It's rare that the difference is significant, however.  In most uses of these
patterns, runtime is dominated by the useful work being done, not by the
dispatching.  The exception is when a structure to be visited contains many
thousands of elements that need virtually no work done to them.  (For example,
if an XML visitor wrote text nodes out unchanged, and the input was mostly text
nodes.) Under such conditions, the time taken by the dispatch mechanism (whether
name-based or adapter-based) would be more visible.





The author has found, however, that in that situation, one can gain more speed
by registering null adapter factories or \function{DOES_NOT_SUPPORT} for the
element types in question.  This shortens the adapter lookup time enough to make
the adaptation overhead competitive with name-based approaches.  But this only
needs to be done when ``trivial'' elements dominate the structures to be
processed, \emph{and} performance is critical.

\begin{seealso}
\seetitle[http://citeseer.nj.nec.com/nordberg96variations.html]{Variations on
the Visitor Pattern}{A critique of the Visitor pattern that raises some of the
same issues we raise here, and with similar solutions.  Reading its C++
examples, however, will increase your appreciation for the simplicity and
modularity that \function{adapt()} offers!}

\seetitle[http://citeseer.nj.nec.com/woolf96null.html]{The Null Object
Pattern}{The original write-up of this handy approach to simplifying
framework code.}
\end{seealso}

\newpage





















\subsubsection{Replacing introspection with Adaptation, Revisited\label{introspect-elim}}

\begin{quotation}
``Potentially-idempotent adapter functions are a honking great
idea -- let's do more of those'', to paraphrase the timbot.

\hfill --- Alex Martelli, on \newsgroup{comp.lang.python}
\end{quotation}

All programs that use GOTO's can be rewritten without GOTOs, using higher-level
constructs for control flow like function calls, \code{while} loops, and so on.
In the same way, type checks -- and even interface checks -- are not essential
in the presence of higher-level control constructs such as adaptation.  Just as
getting rid of GOTO ``spaghetti code'' helped make programs easier to read and
understand, so too can replacing introspection with adaptation.

In section \ref{replintrowadapt}, we listed three common uses for using
type or interface checks (e.g. using \function{isinstance()}):

\begin{itemize}

\item To manually adapt a supplied component to a needed interface

\item To select one of several possible behaviors, based on the kind of
component supplied

\item To select another component, or take some action, using information
about the interfaces supported by the supplied component

\end{itemize}

By now, you've seen enough uses of the \class{protocols} module that it should
be apparent that all three of the above use cases can -- in principle -- be
handled by adaptation.  However, in the course of moving PEAK from using
introspection to adaptation, I ran into some use cases that at first seemed
very difficult to ``adapt''.  However, once I understood how to handle them,
I realized that there was a straightforward approach to refactoring any
introspection use cases I encountered.  Although this approach seems to have
more than one step, in reality they are all variations on the same theme: expose
the hidden interface, then adapt to it.  Here's how you do it:

\begin{enumerate}
\item First, is this just a case of adapting different types to a common
interface?  If yes, then just declare the adapters and use \function{adapt()}
normally.  If the interface isn't explicit or documented, make it so.

\item Is this a case of choosing a behavior, based on the type?  If yes, then
\emph{define the missing interface} that you want to adapt to.  In other words,
code that switches on type to select a behavior, really wants the behavior to be
in the \emph{other} object.  So, there is in effect an ``undocumented implicit
interface'' that the code is adapting the other object to.  Make the interface
explicit and documented, move the code into adapters (or into the other
classes!), and use \function{adapt()}.

\item Is this a case of choosing a behavior or a component based on using
interfaces as metadata?  If so, this is really a special case of \#2.
An example of this use case is where Zope X3 provides UI components based on
what interfaces an object supports.  In this case, the ``undocumented implicit
interface'' is the ability to select an appropriate UI component!  Or perhaps
it's an ability to provide a set of ``tags'' or ``keys'' that can be used to
look up UI components or other things.  You'll have to decide what the real
``essence'' is.  But either way, you make the needed behavior explicit (as an
interface), and then use \function{adapt()}.
\end{enumerate}

Notice that in each case, the code is demonstrably improved. First, there is
more documentation of the \emph{intended} behavior (as opposed to merely the
actual behavior, which might be broken).  Second, there is greater
extensibility, because it isn't necessary to change the code to add more type
cases.  Third, the code is more readable, because the code's purpose is
highlighted, not all the possible variations of its implementation.  In the
words of Tim Peters, ``Explicit is better than implicit.  Simple is better than
complex.  Sparse is better than dense.  Readability counts.''

Now that we've covered how to replace all forms of introspection with
adaptation, I'll readily admit that I still write code that does introspection
when I'm in ``first draft'' mode!  Brevity is the soul of prototyping, and I
don't mind banging out a few quick \code{if isinstance():} checks in order to
figure out what it is I want the code to do.  But then, I refactor, because
I want my code to be...  adaptable!  Chances are good, that you will too.


